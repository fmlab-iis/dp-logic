
\begin{figure}
  \centering
  \begin{subfigure}{.48\columnwidth}
      \[
      \begin{array}{|c|c|c|}
        \hline
        &
        \multicolumn{2}{c|}{output}
        \\
        \hline
        state & Y & N \\
        \hline
        + & \frac{3}{4} = \frac{1}{2} \cdot \frac{1}{2} + \frac{1}{2} \cdot 1 
          & \frac{1}{4} = \frac{1}{2} \cdot \frac{1}{2} + \frac{1}{2} \cdot 0 
        \\
        \hline
        - & \frac{1}{4} = \frac{1}{2} \cdot \frac{1}{2} + \frac{1}{2} \cdot 0 
          & \frac{3}{4} = \frac{1}{2} \cdot \frac{1}{2} + \frac{1}{2} \cdot 1 
        \\
        \hline
      \end{array}
      \]
    \caption{Survey Mechanism}
    \label{figure:2-dp-table}
  \end{subfigure}  
  \hspace{.05\columnwidth}
  \begin{subfigure}{.40\columnwidth}
    \resizebox{\columnwidth}{!}{
    \begin{tikzpicture}[->,>=stealth',shorten >=1pt,auto,node
      distance=2cm,node/.style={circle,draw}]
      \node[node] (p) at ( 0,  .5) { $+$ };
      \node[node] (n) at ( 0, -.5) { $-$ };
      \node[node] (Y) at (-1.5,  0) { $Y$ };
      \node[node] (N) at ( 1.5,  0) { $N$ };

      \path
      (p) edge node [above] { $\frac{3}{4}$ } (Y)
      (p) edge node [above] { $\frac{1}{4}$ } (N)
      (n) edge node [below] { $\frac{1}{4}$ } (Y)
      (n) edge node [below] { $\frac{3}{4}$ } (N)
      ;
      \end{tikzpicture}
    }
    \caption{Corresponding Markov Chain}
    \label{figure:2-dp-mdp}
  \end{subfigure}
  \caption{Survey Mechanism with 2-Differential Privacy}
  \label{figure:2-dp}
\end{figure}

A simple way to design differentially private mechanisms is to add 
random noises. Consider the survey question: have you been diagnosed
with the disease $X$? In order to protect surveyees' privacy, each
answers the question as follows. The surveyee first flips a
coin. If it is tail, she answers the question truthfully. Otherwise, 
she randomly answers \textit{Yes} or \textit{No}
uniformly~\cite{DR:14:AFDP}. Figure~\ref{figure:2-dp} shows the survey
mechanism and its corresponding Markov chain. In the figure, the states
$+$ and $-$ denote the diagnoses are positive or negative
respectively; the states $Y$ and $N$ denote answers to the survey
question. The random noise boosts the probability of answering
\textit{Yes} or \textit{No} to at least $1/4$ regardless of
diagnoses. Inferences on individual diagnosis can be plausibly denied. 

Let us analyze the survey mechanism briefly. The data universe
$\mathcal{X}$ is $\{ +, - \}$. The mechanism $M$ is a randomized
algorithm with inputs in $\mathcal{X}$ and outputs in $\{ Y, N
\}$. For any $x \in \mathcal{X}$, we have $1/4 \leq \Pr[M(x) = Y] \leq
3/4$. Hence for any neighboring $x, x' \in \mathcal{X}$, $\Pr[M(x) =
Y] \leq 3/4 = 3 \cdot 1/4 \leq 3 \Pr[M (x') =
Y]$. Similarly, $\Pr[M (x) = N] \leq 3 \Pr[M (x') = N]$ for any
neighboring $x, x' \in \mathcal{X}$. The survey mechanism is
$(3, 0)$-differentially private. 

More sophisticated differentially private mechanisms are
available. Consider a query 
$f : \mathcal{X}^n \rightarrow \bbfZ^{\geq 0}$. Let $\alpha \in (0, 1)$. 
The \emph{$\alpha$-geometric mechanism} 
outputs $f(x) + Y$ where $Y$ is a random variable with the geometric
distribution~\cite{GRS:09:UUPM,GRS:12:UUPM} :
\[
\Pr[Y = y] = \frac{1 - \alpha}{1 + \alpha}\alpha^{|y|}
\textmd{ for } y \in \bbfZ.
\]
The $\alpha$-geometric mechanism is oblivious since it has the same
output distribution on any inputs $x, x'$ with $f (x) = f
(x')$. Moreover, it is also $(\alpha^{\Delta (f)}, 0)$-differentially
private for any query $f$ with sensitivity $\Delta (f)$. The range of
the mechanism 
however is $\bbfZ$. It may give nonsensical outputs such as
negative integers for queries with positive ranges.

The \emph{truncated $\alpha$-geometric mechanism over $\underline{n}$}
outputs $f (x) + Z$ where $Z$ is a random variable with the following
distribution: 
\[
\Pr[Z = z] =
\left\{
  \begin{array}{ll}
    0 & \textmd{ if } z < - f (x) \\
    \frac{\alpha^{f (x)}}{1 + \alpha} & \textmd{ if } z = -f (x)\\
    \frac{1 - \alpha}{1 + \alpha}\alpha^{|z|} & 
    \textmd{ if } -f (x) < z < n - f (x)\\
    \frac{\alpha^{n - f (x)}}{1 + \alpha} & \textmd{ if } z = n-f (x)\\
    0 & \textmd{ if } z > n - f (x)
  \end{array}
\right.
\]
Note the range of the truncated $\alpha$-geometric mechanism is
$\underline{n}$. The truncated $\alpha$-geometric mechanism is 
again oblivious; it is also $(\alpha^{\Delta (f)}, 0)$-differentially
private for any query $f$ with sensitivity $\Delta (f)$.
The truncated $1/2$-geometric mechanism over $\underline{5}$ is
given in Figure~\ref{figure:geometric-mechanism-table}.

\begin{figure}
  \centering
  \begin{subfigure}{.40\columnwidth}
    \[
    \begin{array}{|c|c|c|c|c|c|c|}
      \hline
      \textmd{input/output} & 0 & 1 & 2 & 3 & 4 & 5\\
      \hline
      0 & {2}/{3}  & {1}/{6} & {1}/{12} & {1}/{24}  & {1}/{48} & {1}/{48}\\
      \hline
      1 & {1}/{3}  & {1}/{3} & {1}/{6} & {1}/{12}  & {1}/{24} & {1}/{24}\\
      \hline
      2 & {1}/{6}  & {1}/{6} & {1}/{3} & {1}/{6}   & {1}/{12} & {1}/{12}\\
      \hline
      3 & {1}/{12} & {1}/{12} & {1}/{6} & {1}/{3}   & {1}/{6} & {1}/{6}\\
      \hline
      4 & {1}/{24} & {1}/{24} & {1}/{12} & {1}/{6}   & {1}/{3} & {1}/{3}\\
      \hline
      5 & {1}/{48} & {1}/{48} & {1}/{24} & {1}/{12} & {1}/{6} & {2}/{3}\\
      \hline
    \end{array}
    \]
    \caption{$1/2$-Geometric Mechanism}
    \label{figure:geometric-mechanism-table}
  \end{subfigure}
  \hspace{.08\columnwidth}
  \begin{subfigure}{.50\columnwidth}
    \centering
  \resizebox{.85\columnwidth}{!}{
    \begin{tikzpicture}[->,>=stealth',shorten >=1pt,auto,node
      distance=2cm,node/.style={circle,draw}]
      \node[node] (i0) at (0,  1) { $in_0$ };
      \node at (0, 0) { $\vdots$ };
      \node at (.8, -.8) { $\vdots$ };
      \node[node] (i5) at (0, -1) { $in_5$ };

      \node[node] (o0) at (4,  2.5) { $out_0$ };
      \node[node] (o1) at (4,  1.5) { $out_1$ };
      \node[node] (o2) at (4,  0.5) { $out_2$ };
      \node[node] (o3) at (4, -0.5) { $out_3$ };
      \node[node] (o4) at (4, -1.5) { $out_4$ };
      \node[node] (o5) at (4, -2.5) { $out_5$ };

      \path
      (i0) edge node [right=30,above=10] { $2/3$ } (o0)
      (i0) edge node [right=30,above=3] { $1/6$ } (o1)
      (i0) edge node [right=30,above=-3] { $1/12$ } (o2)
      (i0) edge node [right=30,below=-5] { $1/24$ } (o3)
      (i0) edge node [right=30,below] { $1/48$ } (o4)
      (i0) edge node [right=30,below=8] { $1/48$ } (o5)

      \hide{
      (i1) edge node [above] {  } (o0)
      (i1) edge node [above] {  } (o2)
      (i1) edge node [above] {  } (o3)
      (i1) edge node [above] {  } (o4)
      (i1) edge node [above] {  } (o5)
      
      (i2) edge node [above] {  } (o0)
      (i2) edge node [above] {  } (o2)
      (i2) edge node [above] {  } (o3)
      (i2) edge node [above] {  } (o4)
      (i2) edge node [above] {  } (o5)
      
      (i3) edge node [above] {  } (o0)
      (i3) edge node [above] {  } (o2)
      (i3) edge node [above] {  } (o3)
      (i3) edge node [above] {  } (o4)
      (i3) edge node [above] {  } (o5)
      
      (i4) edge node [above] {  } (o0)
      (i4) edge node [above] {  } (o2)
      (i4) edge node [above] {  } (o3)
      (i4) edge node [above] {  } (o4)
      (i4) edge node [above] {  } (o5)
      }
      
      (i5) edge node [right=10,above=16] { $1/48$ } (o0)
      \hide{
      (i5) edge node [above] {  } (o2)
      (i5) edge node [above] {  } (o3)
      (i5) edge node [above] {  } (o4)
      }
      (i5) edge node [right=15,below=8] { $2/3$ } (o5)
      ;
      \end{tikzpicture}
    }
    \caption{Markov Chain}
    \label{figure:geometric-mechanism-markov-chain}
  \end{subfigure}
  
  \caption{A Markov Chain for $1/2$-Geometric Mechanism}
  \label{figure:geometric-mechanism}
\end{figure}

Similar to the survey mechanism, it is 
straightfoward to model the truncated $1/2$-geometric
mechanisms as Markov chains. Let the state $in_k$ and $out_l$
denote the input value $k$ and output value $l$ respectively. Define
the state set $S = \{ in_k, out_k : k \in \underline{n} \}$.  
The probability transition $\wp (in_k, out_l)$ is the probability of the
output value $l$ on the input value $k$ as defined in the
mechanism. Figure~\ref{figure:geometric-mechanism-markov-chain} gives 
the Markov chain of the truncated
$1/2$-geometric mechanism over $\underline{5}$.


\begin{figure}
  \centering
  \begin{subfigure}{.40\columnwidth}
    \resizebox{\columnwidth}{!}{
    \begin{tikzpicture}[->,>=stealth',shorten >=1pt,auto,node
      distance=2cm,node/.style={circle,draw}]
      \node[node] (p) at ( 0,  1.5) { $+$ };
      \node[node] (q) at ( 0, -1.5) { $-$ };
      \node[node] (F) at (-1.5,  0) { $Y$ };
      \node[node] (T) at ( 1.5,  0) { $N$ };

      \path
      (p) edge [bend right=45] node [left] { $L, .75$ } (F)
      (p) edge [bend left=45] node [right] { $L, .25$ } (T)
      (p) edge [bend left=45] node [right=-12,above] { $H, .8$ } (F)
      (p) edge [bend right=45] node [right=12,above] { $H, .2$ } (T)

      (q) edge [bend left=45] node [left] { $L, .25$ } (F)
      (q) edge [bend right=45] node [right] { $L, .75$ } (T)
      (q) edge [bend right=45] node [left=12,below] { $H, .2$ } (F)
      (q) edge [bend left=45] node [right=12,below] { $H, .8$ } (T)

      (F) edge [loop left] node [below] { $-, 1$ } (F)
      (T) edge [loop right] node [below] { $-, 1$ } (T)
      ;
      \end{tikzpicture}
    }
    \caption{Markov Decision Process}
    \label{figure:simple-mdp-mdp}
  \end{subfigure}
  \begin{subfigure}{.58\columnwidth}
    \resizebox{\columnwidth}{!}{
    \begin{tikzpicture}[->,>=stealth',shorten >=1pt,auto,node
      distance=2cm,node/.style={circle,draw,inner sep=1pt}]
      \node[node] (pq) at (-2.1, 2.1) { $+-$ };
      \node[node] (qp) at ( 2.1,-2.1) { $-+$ };
      \node[node] (RR) at (-2.1,-2.1) { $YY$ };
      \node[node] (BB) at ( 2.1, 2.1) { $NN$ };
      \node[node] (RB) at (- .7,- .7) { $YN$ };
      \node[node] (BR) at (  .7,  .7) { $NY$ };

      \path
      (pq) edge [bend right=5] node [left=1,below=12,rotate=270] { $L, .1875$ } (RR)
      (pq) edge [bend left=5]  node [above=5,right=1] { $L, .1875$ } (BB)

      (pq) edge [bend left=5]  node [right=4,rotate=270] { $H, .16$ } (RR)
      (pq) edge [bend right=5] node [below=5,right=1] { $H, .16$ } (BB)

      (pq) edge [bend left=5]  node [left,below=10,rotate=295]  { $L, .5625$ } (RB)
      (pq) edge [bend right=5] node [left,below,rotate=330]  { $L, .0625$ } (BR)

      (pq) edge [bend right=5] node [right=2,above,rotate=300] { $H, .64$ } (RB)
      (pq) edge [bend left=5]  node [right,above,rotate=340] { $H, .04$ } (BR)



      (qp) edge [bend right=5] node [above=5,left=1] { $L, .1875$ } (RR)
      (qp) edge [bend left=5]  node [above=30,left=5,rotate=90] { $L, .1875$ } (BB)

      (qp) edge [bend left=5]  node [below=5,left=1] { $H, .16$ } (RR)
      (qp) edge [bend right=5] node [above,right=5,rotate=90] { $H, .16$ } (BB)

      (qp) edge [bend left=5]  node [left,below,rotate=330]  { $L, .0625$ } (RB)
      (qp) edge [bend right=5] node [left=3,below=3,rotate=295]  { $L, .5625$ } (BR)

      (qp) edge [bend right=5] node [left=5,above,rotate=330] { $H, .04$ } (RB)
      (qp) edge [bend left=5]  node [right=1,above=1,rotate=295] { $H, .64$ } (BR)


      (RR) edge [loop left] node [above] { $-, 1$ } (RR)
      (BB) edge [loop right] node [above] { $-, 1$ } (BB)
      (RB) edge [loop below] node [left] { $-, 1$ } (RB)
      (BR) edge [loop above] node [right] { $-, 1$ } (BR)
      ;
      \end{tikzpicture}
\hide{
    \begin{tikzpicture}[->,>=stealth',shorten >=1pt,auto,node
      distance=2cm,node/.style={circle,draw,inner sep=1pt}]
      \node[node] (pq) at (   0,   0) { $+-$ };
      \node[node] (RR) at (-  2,   0) { $YY$ };
      \node[node] (BB) at (   2,   0) { $NN$ };
      \node[node] (RB) at (   0, 1.5) { $YN$ };
      \node[node] (BR) at (   0,-1.5) { $NY$ };

      \path
      (pq) edge [bend right=15] node [above] { $L, .1875$ } (RR)
      (pq) edge [bend left=15]  node [above] { $L, .1875$ } (BB)

      (pq) edge [bend left=15]  node [below] { $H, .16$ } (RR)
      (pq) edge [bend right=15] node [below] { $H, .16$ } (BB)

      (pq) edge [bend left=15]  node [left]  { $L, .5625$ } (RB)
      (pq) edge [bend right=15] node [left]  { $L, .0625$ } (BR)

      (pq) edge [bend right=15] node [right] { $H, .64$ } (RB)
      (pq) edge [bend left=15]  node [right] { $H, .04$ } (BR)

      (RR) edge [loop below] node [below] { $-, 1$ } (RR)
      (BB) edge [loop above] node [above] { $-, 1$ } (BB)
      (RB) edge [loop left] node [left] { $-, 1$ } (RB)
      (BR) edge [loop right] node [right] { $-, 1$ } (BR)
      ;
      \end{tikzpicture}
}
    }
    \caption{Product Markov Decision Process}
    \label{figure:simple-mdp-product}
  \end{subfigure}  
  \caption{Markov Decision Process and its Self-Product}
  \label{figure:simple-mdp}
\end{figure}

Mechanisms are not necessarily closed randomized algorithms; they may
perform different compution on users' requests. 
We use Markov decision processes to model such 
interactive mechanisms. Specifically, external inputs are modeled by
actions. Behaviors associated with different inputs are modeled by
distributions associated with actions.

Consider again the survey mechanism. Suppose we would like to design
an interactive mechanism which adjusts random noises on surveyers'
requests. When the surveyer requests low-accuracy answers, the
surveyee uses the survey mechanism as before. When high-accuracy
answers are requested, the surveyee answers \textit{Yes} with
probability $4/5$ and \textit{No} with probability $1/5$ when she has
positive diagnosis. She answers \textit{Yes} with probability $1/5$
and \textit{No} with probability $4/5$ when she is not
diagnosed with the disease X. This gives an interactive mechanism
corresponding to the Markov decision process shown in
Figure~\ref{figure:simple-mdp-mdp}. 

In the figure, the states $+$,
$-$, $Y$, and $N$ are interpreted as before. The actions $L$ and $H$
denote low- and high-accuracy requests respectively. Let $M_H$ denote
the Markov chain derived by high-accuracy requests.
Observe that $1/5 \leq \Pr[M_H(x) = Y] \leq 4/5$ for any $x
\in \mathcal{X}$. Hence $\Pr[M_H (x) = Y] \leq 4/5 = 4 \cdot 1/5 \leq
4 \Pr[M_H (x') = Y]$ for any neighboring $x, x' \in
\mathcal{X}$. Similarly, $\Pr[M_H (x) = N] \leq 4 \Pr[M_H (x') =
N]$. The high-accuracy mechanism is $(4,0)$-differentially private. 
The privacy guarantees vary from accuracy requests.