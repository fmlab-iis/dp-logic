
% privacy and differential privacy

In the era of data analysis, personal information is constantly
collected and analyzed by various parties. Privacy has become an
important issue for every individual. In order to address such
concerns, the research community has proposed several privacy
preserving mechanisms over the years (see~\cite{FWC:10:PPDP} for a
slightly outdated survey). Among these mechanisms, differential
privacy has attracted much attention from theoretical computer science
to industry~\cite{DR:14:AFDP,JLE:14:DPML,A:16:EPYU}.

Differential privacy formalizes the tradeoff between privacy and
utility in data analysis. Intuitively, a randomized data analysis
mechanism is differentially private if it behaves similarly on similar
input databases~\cite{DMNS:06:CNSPD,D:06:DP}. Consider, for example,
the Laplace mechanism where a random noise is added to analysis
results by Laplace distribution~\cite{DR:14:AFDP}. Random noises
hide the differences of analysis results from different databases
provided similar databases have similar results. Clearly, more 
randomness gives more privacy but less utility in released noisy
results. Under the framework of differential privacy, data analysts
can balance the tradeoff rigorously in their data analysis
mechanisms~\cite{DR:14:AFDP,JLE:14:DPML}.

% model checking differentially private implementation

Designing differentially private mechanisms can be tedious for
sophisticated data analyses. Privacy leak has also been observed from
data analysis programs with differentially private
designs~\cite{M:12:SLSBDP,TKBW:17:PLAI}. This calls for formal
analysis of differential privacy on both designs and implementations. 
In tis paper, we propose the logic $\dpCTL$ for specifying
differentially private states and develop their model checking
algorithms. Data analysts can automatically verify their designs and
implementations with our techniques. Most interestingly, our
techniques can be adopted rather easily by existing probabilistic
model checkers. Hopefully, more interaction between model checking and
privacy analysis will follow in the future.

% Markov chains and MDP as formal models, actions made by adversary

In differential privacy, data analysis mechanisms are
but randomized algorithms. We follow the standard practice in
probabilistic model checking to formalize such mechanisms by
by Markov chains or Markov decision processes. 
When a data analysis mechanism does not
interact with its environment, it is formalized as a Markov
chain. Otherwise, its interactions are formalized by actions in
Markov decision processes. Our formalization effectively assumes that
actions are controlled by adversaries. It is subsequently necessary to
consider attacks from all possible adversaries in order to establish
differential privacy.
%\todo{What is a state?}

% new differential privacy state modal operator

Two ingredients are introduced to specify differentially private
behaviors. A reflexive and commutative user-defined binary relation
over states is required to formalize similar states. Two states are
similar if they are related by the binary relation. We moreover add
the path quantifier $\dpriv{\epsilon}{\delta}$ for similar
behaviors. Informally, a state satisfies $\dpriv{z\epsilon}{\delta}
\phi$ if it has probabilistically similar path behavior $\phi$ as
its similar states. Consider, for instance, a binary data analysis
mechanism computing the likelihood ($\mathit{high}$ or $\mathit{low}$)
of an epidemic in a population. Then a state satisfies
$\dpriv{\epsilon}{\delta} (\F \mathit{high}) \wedge
\dpriv{\epsilon}{\delta} (\F \mathit{low})$ if the mechanism is
$(\epsilon, \delta)$-differentially private at the state.
\todo{not clear here}

% differential privacy of infinite behaviors

An important byproduct of this work is to analyze privacy over
infinite behaviors. Consider the formula $\dpriv{\epsilon}{\delta} 
(\G \PJ{[1,1]} (\F \mathit{high}))$. A state satisfies the formula if it
is probabilistically similar to always almost surely
answer $\mathit{high}$ as similar states. Compared to the unbounded
but finite behavior specified by $\F \mathit{high}$, 
$\G \PJ{[1,1]} (\F \mathit{high})$ specifies an infinite behavior.
Not only can differential privacy specify randomized algorithms, it
also naturally generalize to reactive systems.
Privacy analysis applies to systems with infinite behaviors as well.

% construction and complexity

In order to verify $\dpCTL$ properties, we extend the standard
probabilistic model checking algorithms. For Markov chains, the states
satisfying a subformula $\dpriv{\epsilon}{\delta} \phi$ are computed
by a simple variant of the algorithm for the subformula
$\PJ{J} \phi$. For Markov Decision Processes, a product construction
followed by the standard algorithm for infinite horizon expected total
undiscounted reward for Markov Decision Processes with absorbing
states will do. The time complexity of our model checking algorithm is
still polynomial. The logic $\dpCTL$ essentially obtains its
expressiveness for free.

% analyze mechanism


