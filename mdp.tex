%\subsection{Markov Decision Processes}
\begin{wrapfigure}[14]{r}{.45\columnwidth}
%\begin{figure}
  \centering
    \resizebox{.45\columnwidth}{!}{
    \begin{tikzpicture}[->,>=stealth',shorten >=1pt,auto,node
      distance=2cm,node/.style={circle,draw}]
      \node[node] (p) at ( 0,  1.5) { $+$ };
      \node[node] (q) at ( 0, -1.5) { $-$ };
      \node[node] (F) at (-1.5,  0) { $s$ };
      \node at (-2, .5) { $\mathit{out}_Y$ };
      \node[node] (T) at ( 1.5,  0) { $t$ };
      \node at ( 2, .5) { $\mathit{out}_N$ };

      \path
      (p) edge [bend right=45] node [left] { $L, .75$ } (F)
      (p) edge [bend left=45] node [right] { $L, .25$ } (T)
      (p) edge [bend left=45] node [right=-12,above] { $H, .8$ } (F)
      (p) edge [bend right=45] node [right=12,above] { $H, .2$ } (T)

      (q) edge [bend left=45] node [left] { $L, .25$ } (F)
      (q) edge [bend right=45] node [right] { $L, .75$ } (T)
      (q) edge [bend right=45] node [left=12,below] { $H, .2$ } (F)
      (q) edge [bend left=45] node [right=12,below] { $H, .8$ } (T)

%      (F) edge [loop left] node [below] { $-, 1$ } (F)
%      (T) edge [loop right] node [below] { $-, 1$ } (T)
      ;
      \end{tikzpicture}
    }
    \caption{Markov Decision Process}
    \label{figure:simple-mdp}
\end{wrapfigure}
%\end{figure}

\hide{Mechanisms are not necessarily closed randomized algorithms; they may
perform different computation on users' requests.
We use Markov decision processes to model such
interactive mechanisms. Specifically, external inputs are modeled by
actions. Behaviors associated with different inputs are modeled by
distributions associated with actions.
}

We use MDPs to model 
interactive mechanisms. Specifically, external inputs are modeled by
actions. Behaviors associated with different inputs are modeled by
distributions associated with actions.


Consider again the survey mechanism. Suppose we would like to design
an interactive mechanism which adjusts random noises on surveyors'
requests. When the surveyor requests low-accuracy answers, the
surveyee uses the survey mechanism as before. When high-accuracy
answers are requested, the surveyee answers \textit{Yes} with
probability $\frac{4}{5}$ and \textit{No} with probability $\frac{1}{5}$ when she has
positive diagnosis. She answers \textit{Yes} with probability $\frac{1}{5}$
and \textit{No} with probability $\frac{4}{5}$ when she is not
diagnosed with the disease X. This gives an interactive mechanism
corresponding to the MDP shown in
Figure~\ref{figure:simple-mdp}.

In the figure, the states $+$,
$-$, $s$, and $t$ are interpreted as before. The actions $L$ and $H$
denote low- and high-accuracy requests respectively. Let $M_H$ denote
the Markov chain derived by high-accuracy requests.
Note that the high-accuracy mechanism is $(\ln 4,0)$-differentially private.
Unlike non-interactive mechanisms, the privacy guarantees vary from accuracy requests.

\subsection{Semantics}
The logic $\dpCTLstar$ can be interpreted over MDPs. Semantics over Markov chains generalizes to MDPs as well. %Note that two states with different enabled actions
%are trivially distinguishable; no privacy can be preserved.
Let $M = (S, \Act, \wp, L)$ be an MDP. Define the satisfaction
relation $M, \neighbor{S}, s \models \Phi$ for $\PJ{J}$ and $\dpriv{\epsilon}{\delta} \phi$ (others are standard) as follows.
\begin{eqnarray*}
%  M, \neighbor{S}, s \models \top\\
%  M, \neighbor{S}, s \models p  & \textmd{ if } &  p \in L(s)\\
%  M, \neighbor{S}, s \models \neg \Phi  & \textmd{ if } &  M, \neighbor{S}, s \not\models \Phi\\
%  M, \neighbor{S}, s \models \Phi_0 \wedge \Phi_1  & \textmd{ if } &  M, \neighbor{S}, s \models \Phi_0 \textmd{ and }
%  M, \neighbor{S}, s \models \Phi_1\\
  M, \neighbor{S}, s \models \PJ{J} \phi
  & \textmd{ if } &
  \myPr{s}{M_{\scheduler{S}}}{\neighbor{S}}{\phi} \in J
  \textmd{ for every scheduler } \scheduler{S}\\
  M, \neighbor{S}, s \models \dpriv{\epsilon}{\delta} \phi
  & \textmd{ if } &
  \textmd{for all } t \textmd{ with } s \neighbor{S} t \textmd{ and
   query scheduler } \scheduler{Q},
   \myPr{s}{M_{\scheduler{Q}}}{\neighbor{S}}{\phi} \leq
  e^{\epsilon} \cdot
   \\
  & &
   \myPr{t}{M_{\scheduler{Q}}}{\neighbor{S}}{\phi} + \delta
   \textmd{ and }
   \myPr{t}{M_{\scheduler{Q}}}{\neighbor{S}}{\phi} \leq e^{\epsilon} \cdot
   \myPr{s}{M_{\scheduler{Q}}}{\neighbor{S}}{\phi} + \delta
\end{eqnarray*}

Recall that $M_{\scheduler{S}}$ is but a Markov chain. The semantics
of $M_{\scheduler{S}}, \neighbor{S}, \pi \models \phi$ and hence the
probability $\myPr{s}{M_{\scheduler{S}}}{\neighbor{S}}{\phi}$ are
defined as in Markov chains.
The semantics of $\dpCTLstar$ on MDPs
is again standard except the differentially private operator
$\dpriv{\epsilon}{\delta}$. For any path formula $\phi$,
$\dpriv{\epsilon}{\delta} \phi$ specifies states whose probability
of having paths satisfying $\phi$ are $(\epsilon, \delta)$-close to
those of all its neighbors for query schedulers. That is, no
query scheduler can force any of neighbors to have
probabilistically distinguishable behaviors.

\noindent
\paragraph{Justification of query schedulers.}
We use query schedulers in the semantics for the differentially
private operator. A definition with history-dependent schedulers
might be
\begin{eqnarray*}
  M, \neighbor{S}, s \models \dpriv{\epsilon}{\delta}^{\mathit{bad}} \phi
  & \textmd{ if } &
  \textmd{for all } t \textmd{ with } s \neighbor{S} t \textmd{ and
  scheduler } \scheduler{S},
  \myPr{s}{M_{\scheduler{S}}}{\neighbor{S}}{\phi} \leq
  e^{\epsilon} \cdot\\
  && \myPr{t}{M_{\scheduler{S}}}{\neighbor{S}}{\phi} + \delta \textmd{ and }
  \myPr{t}{M_{\scheduler{S}}}{\neighbor{S}}{\phi} \leq
  e^{\epsilon} \cdot \myPr{s}{M_{\scheduler{S}}}{\neighbor{S}}{\phi}
  + \delta
\end{eqnarray*}
A state satisfies $\dpriv{\epsilon}{\delta}^{\mathit{bad}} \phi$ if
no history-dependent scheduler could differentiate the probabilities
of haveing paths satisfying
$\phi$ from neighbors. Recall that a history-dependent scheduler takes
actions by previous states. Such a definition would allow
schedulers to take different actions from different states. 
Two neighbors could be differentiated by different action sequences
erroneously.
A query scheduler $\scheduler{Q} : \bbfZ^{\geq 0} \rightarrow \Act$,
on the other hand, corresponds to a query sequence. A state satisfies
$\dpriv{\epsilon}{\delta} \phi$ if no query sequence can differentiate
the probabilities of having paths satisfying $\phi$ from neighbors.
Our semantics agrees with the informal interpretation of differential
privacy for such systems. We hence
prefer the original definition.

\subsection{The Model Checking Problem}
For an MDP $M =
(S, \Act, \wp, L)$, we want to compute $\{ M, \neighbor{S}, s \models
\dpriv{\epsilon}{\delta} \phi \}$ . Recall the semantics of $\dpriv{\epsilon}{\delta}
\phi$. Given $s, t$ with $s \neighbor{S} t$ and a path formula
$\phi$, we need to decide whether
$\myPr{s}{M_{\scheduler{Q}}}{\neighbor{S}}{\phi} \leq
e^{\epsilon} \myPr{t}{M_{\scheduler{Q}}}{\neighbor{S}}{\phi} + \delta$
for every query scheduler $\scheduler{Q}$.
If the path formula only contains next operators, it is easy to deal with. When $\phi$ is $\gX B$ with $B\subseteq S$, only one step behavior needs to be considered. Thus only the first action in the query sequence is needed. It can also be easily generalized to nested next operators:  one needs only to enumerate
all actions query sequences of length obtained by the number of nested $\X$ operators.
The problem however is undecidable in general.

\begin{theorem}\label{theorem:mdp-model-checking}
The $\dpCTLstar$ model checking problem for MDPs  is undecidable.
\end{theorem}

The proof is put in Appendix. We discuss some decidable special cases. Consider the formula $\phi:=F B$ with $B\subseteq S$ and assume that states in $B$ are absorbing (with only self-loops). For the case $\epsilon=0$, the condition reduces to
$\myPr{s}{M_{\scheduler{Q}}}{\neighbor{S}}{F B} -
\myPr{t}{M_{\scheduler{Q}}}{\neighbor{S}}{F B}\le \delta$. If $\delta=0$ it is the classical language equivalence problem for probabilistic automata~\cite{Rabin63}, which can be solved in polynomial time. However, if $\delta>0$, the problem
becomes an approximate version of the language equivalence problem. To the best of our knowledge, the decision problem is still open, and has only be investigated for the special case when all states are connected~\cite{Tzeng92}.

\hide{
Since the scheduler attaining the maximally probable
behavior from a state may be different from the scheduler attaining
the maximally probable behavior from its neighbors, two
neighbors may still be distinguished by a scheduler. The
weaker definition does not preserve differential privacy. We hence
prefer the original definition.
}
