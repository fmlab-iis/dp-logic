%\subsection{Markov Decision Processes}
\begin{wrapfigure}[14]{r}{.45\columnwidth}
%\begin{figure}
  \centering
    \resizebox{.45\columnwidth}{!}{
    \begin{tikzpicture}[->,>=stealth',shorten >=1pt,auto,node
      distance=2cm,node/.style={circle,draw}]
      \node[node] (p) at ( 0,  1.5) { $+$ };
      \node[node] (q) at ( 0, -1.5) { $-$ };
      \node[node] (F) at (-1.5,  0) { $s$ };
      \node at (-2, .5) { $\mathit{out}_Y$ };
      \node[node] (T) at ( 1.5,  0) { $t$ };
      \node at ( 2, .5) { $\mathit{out}_N$ };

      \path
      (p) edge [bend right=45] node [left] { $L, .75$ } (F)
      (p) edge [bend left=45] node [right] { $L, .25$ } (T)
      (p) edge [bend left=45] node [right=-12,above] { $H, .8$ } (F)
      (p) edge [bend right=45] node [right=12,above] { $H, .2$ } (T)

      (q) edge [bend left=45] node [left] { $L, .25$ } (F)
      (q) edge [bend right=45] node [right] { $L, .75$ } (T)
      (q) edge [bend right=45] node [left=12,below] { $H, .2$ } (F)
      (q) edge [bend left=45] node [right=12,below] { $H, .8$ } (T)

%      (F) edge [loop left] node [below] { $-, 1$ } (F)
%      (T) edge [loop right] node [below] { $-, 1$ } (T)
      ;
      \end{tikzpicture}
    }
    \caption{Markov Decision Process}
    \label{figure:simple-mdp}
\end{wrapfigure}
%\end{figure}

\hide{Mechanisms are not necessarily closed randomized algorithms; they may
perform different computation on users' requests.
We use Markov decision processes to model such
interactive mechanisms. Specifically, external inputs are modeled by
actions. Behaviors associated with different inputs are modeled by
distributions associated with actions.
}

We use MDPs to model 
interactive mechanisms. Specifically, external inputs are modeled by
actions. Behaviors associated with different inputs are modeled by
distributions associated with actions.


Consider again the survey mechanism. Suppose we would like to design
an interactive mechanism which adjusts random noises on surveyors'
requests. When the surveyor requests low-accuracy answers, the
surveyee uses the survey mechanism as before. When high-accuracy
answers are requested, the surveyee answers \textit{Yes} with
probability $\frac{4}{5}$ and \textit{No} with probability $\frac{1}{5}$ when she has
positive diagnosis. She answers \textit{Yes} with probability $\frac{1}{5}$
and \textit{No} with probability $\frac{4}{5}$ when she is not
diagnosed with the disease X. This gives an interactive mechanism
corresponding to the MDP shown in
Figure~\ref{figure:simple-mdp}.

In the figure, the states $+$,
$-$, $s$, and $t$ are interpreted as before. The actions $L$ and $H$
denote low- and high-accuracy requests respectively. Let $M_H$ denote
the Markov chain derived by high-accuracy requests.
Note that the high-accuracy mechanism is $(\ln 4,0)$-differentially private.
Unlike non-interactive mechanisms, the privacy guarantees vary from accuracy requests.

\subsection{Above Threshold Mechanism}
\label{subsec:threshold}

In differential privacy, an offline mechanism releases outputs only
once and plays no further role; an online (interactive) mechanism
allows analysts to ask queries adaptively based on previous
responses. The mechanisms we constructed previously are offline
mechanisms. On the other hand, online mechanisms permit further queries. 
To maintain
privacy, online mechanisms may decide to release
information differently. For instance, they may disable further
queries after a certain output is disclosed.

Below we describe an online mechanism adapted from~\cite{DR:14:AFDP}.
Given a threshold and a series of adaptive queries, we only care for
the queries whose results are above the threshold; queries below the
threshold only disclose minimal information and hence are considered
irrelevant. Since query results above the threshold convey useful
information, let us assume the mechanism will halt on the first such
query result for simplicity. The question is to design a
privacy-preserving mechanism to have similar behaviors on similar
databases. In~\cite{DR:14:AFDP}, a mechanism is design for continuous
queries by applying the Laplace mechanism. We will develop a mechanism
for bounded and discrete queries using the geometric mechanism.

\begin{algorithm}
	\begin{algorithmic}[1]
		\Procedure{AboveThreshold}{$d$, $\{ f_1, f_2, \ldots
                  \}$, $t$}
		\Match{$t$}
		\Comment{obtain $t'$ by $\frac{1}{4}$-geometric mechanism}
		\lCase{$0$}{$t' \leftarrow 0, 1,2,3,4,5$ with probability
			$\frac{4}{5}$,$\frac{3}{20}$,
			$\frac{3}{80}$,$\frac{3}{320}$,
			$\frac{3}{1280}$,$\frac{1}{1280}$ respectively}
		\lCase{$1$}{$t' \leftarrow 0, 1,2,3,4,5$ with probability
			$\frac{1}{5}$,$\frac{3}{5}$,
			$\frac{3}{20}$,$\frac{3}{80}$,
			$\frac{3}{320}$,$\frac{1}{320}$ respectively}
		\lCase{$2$}{$t' \leftarrow 0, 1,2,3,4,5$ with probability
			$\frac{1}{20}$,$\frac{3}{20}$,
			$\frac{3}{5}$,$\frac{3}{20}$,
			$\frac{3}{80}$,$\frac{1}{80}$ respectively}
		\lCase{$3$}{$t' \leftarrow 0, 1,2,3,4,5$ with probability
			$\frac{1}{80}$,$\frac{3}{80}$,
			$\frac{3}{20}$,$\frac{3}{5}$,
			$\frac{3}{20}$,$\frac{1}{20}$ respectively}
		\lCase{$4$}{$t' \leftarrow 0, 1,2,3,4,5$ with probability
			$\frac{1}{320}$,$\frac{3}{320}$,
			$\frac{3}{80}$,$\frac{3}{20}$,
			$\frac{3}{5}$,$\frac{1}{5}$ respectively}
		\lCase{$5$}{$t' \leftarrow 0, 1,2,3,4,5$ with probability
			$\frac{1}{1280}$,$\frac{3}{1280}$,
			$\frac{3}{320}$,$\frac{3}{80}$,
			$\frac{3}{20}$,$\frac{4}{5}$ respectively}
		\EndMatch
		\For{each query $f_i$}
                \State{$r_i \leftarrow f_i (d)$}
		\Match{ $r_i$}
		\Comment{obtain $r_{i}'$ by $\frac{1}{2}$-geometric mechanism}
		\lCase{$0$}{$r_{i}' \leftarrow 0, 1,2,3,4,5$ with probability
			$\frac{2}{3}$,$\frac{1}{6}$,
			$\frac{1}{12}$,$\frac{1}{24}$,
			$\frac{1}{48}$,$\frac{1}{48}$ respectively}
		\lCase{$1$}{$r_{i}' \leftarrow 0, 1,2,3,4,5$ with probability
			$\frac{1}{3}$,$\frac{1}{3}$,
			$\frac{1}{6}$,$\frac{1}{12}$,
			$\frac{1}{24}$,$\frac{1}{24}$ respectively}
		\lCase{$2$}{$r_{i}' \leftarrow 0, 1,2,3,4,5$ with probability
			$\frac{1}{6}$,$\frac{1}{6}$,
			$\frac{1}{3}$,$\frac{1}{6}$,
			$\frac{1}{12}$,$\frac{1}{12}$ respectively}
		\lCase{$3$}{$r_{i}' \leftarrow 0, 1,2,3,4,5$ with probability
			$\frac{1}{12}$,$\frac{1}{12}$,
			$\frac{1}{6}$,$\frac{1}{3}$,
			$\frac{1}{6}$,$\frac{1}{6}$ respectively}
		\lCase{$4$}{$r_{i}' \leftarrow 0, 1,2,3,4,5$ with probability
			$\frac{1}{24}$,$\frac{1}{24}$,
			$\frac{1}{12}$,$\frac{1}{6}$,
			$\frac{1}{3}$,$\frac{1}{3}$ respectively}
		\lCase{$5$}{$r_{i}' \leftarrow 0, 1,2,3,4,5$ with probability
			$\frac{1}{48}$,$\frac{1}{48}$,
			$\frac{1}{24}$,$\frac{1}{12}$,
			$\frac{1}{6}$,$\frac{2}{3}$ respectively}
		\EndMatch
                \State{\textbf{if} $r'_i \geq t'$ \textbf{then} 
                  \textbf{halt} with $a_i = \top$ \textbf{else}
                  $a_i = \bot$}
\hide{
		\If{$r_{i}'>t'$}
		\State{$a_i=\top$}
		\State{\textbf{halt}}
		\Else
		\State{$a_i=\bot$}
		\EndIf
}
		\EndFor
		\EndProcedure
		
	\end{algorithmic}
	\caption{Input: private database $d$, 
          queries $f_i : \{ 0, 1, \ldots, 5 \} \rightarrow \{ 0, 1,
          \ldots, 5 \}$ with sensitivity $1$, threshold $t \in \{ 0,
          1, \ldots, 5 \}$; Output: $a_1, a_2, \ldots$}
	\label{algorithm:online-model}
\end{algorithm}

Consider a threshold $t \in
\{ 0, 1 \ldots, 5\}$, and queries $\{ f_i : f_i $ with sensitivity $\Delta
(f_i) = 1 \}$. In order to protect privacy, our mechanism applies the
truncated $\frac{1}{4}$-geometric mechanism to obtain a noisy
threshold $t'$. For each query, the truncated $\frac{1}{2}$-geometric
mechanism is applied to its result. If the noisy result is not less
than the noisy threshold, our mechanism outputs $\top$ and
halts. Otherwise, it outputs $\bot$ and continues to the next query
(Algorithm~\ref{algorithm:online-model}). The above threshold
mechanism outputs a sequence of the form $\bot^* \top$. On
similar databases, we would like to show that the discrete above
threshold mechanism outputs the same sequence with similar
probabilities. 

\todo{explain Markov chain}

\begin{figure}
  \centering
    \resizebox{.8\columnwidth}{!}{
    \begin{tikzpicture}[->,>=stealth',shorten >=1pt,auto,node
      distance=2cm,node/.style={circle,draw}]
      \node[node] (t0r1) at (0, 4.25) { $t_0r_1$ };

      \draw (2.9, 3.45) rectangle (4.9, .05);
      \node at (4.65, 2.75) { $\cdots $};
      \node[node] (t'0r1) at (3.9, 2.75) { $t'_0r_1$ };
      \node at (3.2, 2.75) { $\cdots $};
      \draw (-2.8, 3.45) rectangle (2.8, .05);
      \node[node] (t'1r0) at (2.1, 2.75) { $t'_1r_0$ };
      \node[node] (t'1r1) at (0, 2.75) { $t'_1r_1$ };
      \node[node] (t'1r2) at (-2.1, 2.75) { $t'_1r_2$ };
      \node at (0, 1.85) { $\vdots$ };
      \node at (-2.1, 1.85) { $\vdots$ };
      \draw (-2.9, 3.45) rectangle (-4.9, .05);
      \node at (-3.15, 2.75) { $\cdots$ };
      \node[node] (t'2r1) at (-3.9, 2.75) { $t'_2r_1$ };
      \node at (-4.6, 2.75) { $\cdots$ };

      \node[node] (t'1r'0) at (2.1, .75) { $t'_1r'_0$ };
      \node[node] (t'1r'1) at (0, .75) { $t'_1r'_1$ };
      \node[node] (t'1r'2) at (-2.1, .75) { $t'_1r'_2$ };

      \node[node] (tt0rr0) at (2.1, -4.25) { $\underline{t}_0\underline{r}_0$ };

      \draw (2.9, -3.45) rectangle (4.9, -.05);
      \node[node] (tt'0rr0) at (4.2, -2.75) { $\underline{t}'_0\underline{r}_0$ };
      \node at (3.35, -2.75) { $\cdots$ };
      \draw (.8, -3.45) rectangle (2.8, -.05);
      \node[node] (tt'1rr0) at (2.1, -2.75) { $\underline{t}'_1\underline{r}_0$ };
      \node at (1.25, -2.75) { $\cdots$ };

      \draw (.7, -3.45) rectangle (-4.9, -.05);
      \node[node] (tt'2rr0) at (0, -2.75) { $\underline{t}'_2\underline{r}_0$ };
      \node[node] (tt'2rr1) at (-2.1, -2.75) { $\underline{t}'_2\underline{r}_1$ };
      \node[node] (tt'2rr2) at (-4.2, -2.75) { $\underline{t}'_2\underline{r}_2$ };

      \node at (0, -1.65) { $\vdots$ };
      
      \node[node] (tt'2rr'0) at (0, -.75) { $\underline{t}'_2\underline{r}'_0$ };
      \node[node] (tt'2rr'1) at (-2.1, -.75) { $\underline{t}'_2\underline{r}'_1$ };
      \node at (-4.2, -1.65) { $\vdots$ };

      \node[node] (tt'2rr'2) at (-4.2, -.75) { $\underline{t}'_2\underline{r}'_2$ };


      \path
      (t0r1) edge node [above] { $\frac{1}{1}$ } (t'0r1)
      (t0r1) edge node [left] { $\frac{1}{1}$ } (t'1r1)
      (t0r1) edge node [above] { $\frac{1}{1}$ } (t'2r1)

%      (t'1r1) edge node [above] { $\frac{1}{1}$ } (t'1r'0)
%      (t'1r1) edge node [above] { $\frac{1}{1}$ } (t'1r'1)
%      (t'1r1) edge node [above] { $\frac{1}{1}$ } (t'1r'2)

      (t'1r'0) edge [double] node [above]
              { \hide{$0|0, 0|1$} } (t'1r0)
      (t'1r'0) edge [double] node [above] 
              { \hide{$1|0, 1|1, 1|2$} } (t'1r1)
      (t'1r'0) edge [double] node [below] 
              { \hide{$2|1, 2|2$} } (t'1r2)

      (tt0rr0) edge node [below] { $\frac{1}{1}$ } (tt'0rr0)
      (tt0rr0) edge node [left] { $\frac{1}{1}$ } (tt'1rr0)
      (tt0rr0) edge node [below] { $\frac{1}{1}$ } (tt'2rr0)

%      (tt'2rr'0) edge [double] node [above] { $0|0, 1|0$ } (tt'2rr0)
%      (tt'2rr'0) edge [double] node [above] { $0|1, 1|1, 2|1$ } (tt'2rr1)
%      (tt'2rr'0) edge [double] node [above] { $1|2, 2|2$ } (tt'2rr2)

      (tt'2rr'1) edge [double] node [above]
                { \hide{$0|0, 1|0$} } (tt'2rr0)
      (tt'2rr'1) edge [double] node [above]
                { \hide{$0|1, 1|1, 2|1$} } (tt'2rr1)
      (tt'2rr'1) edge [double] node [above]
                { \hide{$1|2, 2|2$} } (tt'2rr2)

      ;
      \end{tikzpicture}
    }
  \caption{Markov Decision Process for Above Threshold}
  \label{figure:mdp-above-threshold}
\end{figure}




\subsection{Semantics}
The logic $\dpCTLstar$ can be interpreted over MDPs. Semantics over Markov chains generalizes to MDPs as well. %Note that two states with different enabled actions
%are trivially distinguishable; no privacy can be preserved.
Let $M = (S, \Act, \wp, L)$ be an MDP. Define the satisfaction
relation $M, \neighbor{S}, s \models \Phi$ for $\PJ{J}$ and $\dpriv{\epsilon}{\delta} \phi$ (others are standard) as follows.
\begin{eqnarray*}
%  M, \neighbor{S}, s \models \top\\
%  M, \neighbor{S}, s \models p  & \textmd{ if } &  p \in L(s)\\
%  M, \neighbor{S}, s \models \neg \Phi  & \textmd{ if } &  M, \neighbor{S}, s \not\models \Phi\\
%  M, \neighbor{S}, s \models \Phi_0 \wedge \Phi_1  & \textmd{ if } &  M, \neighbor{S}, s \models \Phi_0 \textmd{ and }
%  M, \neighbor{S}, s \models \Phi_1\\
  M, \neighbor{S}, s \models \PJ{J} \phi
  & \textmd{ if } &
  \myPr{s}{M_{\scheduler{S}}}{\neighbor{S}}{\phi} \in J
  \textmd{ for every scheduler } \scheduler{S}\\
  M, \neighbor{S}, s \models \dpriv{\epsilon}{\delta} \phi
  & \textmd{ if } &
  \textmd{for all } t \textmd{ with } s \neighbor{S} t \textmd{ and
   query scheduler } \scheduler{Q},
   \myPr{s}{M_{\scheduler{Q}}}{\neighbor{S}}{\phi} \leq
  e^{\epsilon} \cdot
   \\
  & &
   \myPr{t}{M_{\scheduler{Q}}}{\neighbor{S}}{\phi} + \delta
   \textmd{ and }
   \myPr{t}{M_{\scheduler{Q}}}{\neighbor{S}}{\phi} \leq e^{\epsilon} \cdot
   \myPr{s}{M_{\scheduler{Q}}}{\neighbor{S}}{\phi} + \delta
\end{eqnarray*}

Recall that $M_{\scheduler{S}}$ is but a Markov chain. The semantics
of $M_{\scheduler{S}}, \neighbor{S}, \pi \models \phi$ and hence the
probability $\myPr{s}{M_{\scheduler{S}}}{\neighbor{S}}{\phi}$ are
defined as in Markov chains.
The semantics of $\dpCTLstar$ on MDPs
is again standard except the differentially private operator
$\dpriv{\epsilon}{\delta}$. For any path formula $\phi$,
$\dpriv{\epsilon}{\delta} \phi$ specifies states whose probability
of having paths satisfying $\phi$ are $(\epsilon, \delta)$-close to
those of all its neighbors for query schedulers. That is, no
query scheduler can force any of neighbors to have
probabilistically distinguishable behaviors.

\noindent
\paragraph{Justification of query schedulers.}
We use query schedulers in the semantics for the differentially
private operator. A definition with history-dependent schedulers
might be
\begin{eqnarray*}
  M, \neighbor{S}, s \models \dpriv{\epsilon}{\delta}^{\mathit{bad}} \phi
  & \textmd{ if } &
  \textmd{for all } t \textmd{ with } s \neighbor{S} t \textmd{ and
  scheduler } \scheduler{S},
  \myPr{s}{M_{\scheduler{S}}}{\neighbor{S}}{\phi} \leq
  e^{\epsilon} \cdot\\
  && \myPr{t}{M_{\scheduler{S}}}{\neighbor{S}}{\phi} + \delta \textmd{ and }
  \myPr{t}{M_{\scheduler{S}}}{\neighbor{S}}{\phi} \leq
  e^{\epsilon} \cdot \myPr{s}{M_{\scheduler{S}}}{\neighbor{S}}{\phi}
  + \delta
\end{eqnarray*}
A state satisfies $\dpriv{\epsilon}{\delta}^{\mathit{bad}} \phi$ if
no history-dependent scheduler could differentiate the probabilities
of haveing paths satisfying
$\phi$ from neighbors. Recall that a history-dependent scheduler takes
actions by previous states. Such a definition would allow
schedulers to take different actions from different states. 
Two neighbors could be differentiated by different action sequences
erroneously.
A query scheduler $\scheduler{Q} : \bbfZ^{\geq 0} \rightarrow \Act$,
on the other hand, corresponds to a query sequence. A state satisfies
$\dpriv{\epsilon}{\delta} \phi$ if no query sequence can differentiate
the probabilities of having paths satisfying $\phi$ from neighbors.
Our semantics agrees with the informal interpretation of differential
privacy for such systems. We hence
prefer the original definition.

\subsection{The Model Checking Problem}
For an MDP $M =
(S, \Act, \wp, L)$, we want to compute $\{ M, \neighbor{S}, s \models
\dpriv{\epsilon}{\delta} \phi \}$ . Recall the semantics of $\dpriv{\epsilon}{\delta}
\phi$. Given $s, t$ with $s \neighbor{S} t$ and a path formula
$\phi$, we need to decide whether
$\myPr{s}{M_{\scheduler{Q}}}{\neighbor{S}}{\phi} \leq
e^{\epsilon} \myPr{t}{M_{\scheduler{Q}}}{\neighbor{S}}{\phi} + \delta$
for every query scheduler $\scheduler{Q}$.
If the path formula only contains next operators, it is easy to deal with. When $\phi$ is $\gX B$ with $B\subseteq S$, only one step behavior needs to be considered. Thus only the first action in the query sequence is needed. It can also be easily generalized to nested next operators:  one needs only to enumerate
all actions query sequences of length obtained by the number of nested $\X$ operators.
The problem however is undecidable in general.

\begin{theorem}\label{theorem:mdp-model-checking}
The $\dpCTLstar$ model checking problem for MDPs  is undecidable.
\end{theorem}

The proof is put in Appendix. We discuss some decidable special cases. Consider the formula $\phi:=F B$ with $B\subseteq S$ and assume that states in $B$ are absorbing (with only self-loops). For the case $\epsilon=0$, the condition reduces to
$\myPr{s}{M_{\scheduler{Q}}}{\neighbor{S}}{F B} -
\myPr{t}{M_{\scheduler{Q}}}{\neighbor{S}}{F B}\le \delta$. If $\delta=0$ it is the classical language equivalence problem for probabilistic automata~\cite{Rabin63}, which can be solved in polynomial time. However, if $\delta>0$, the problem
becomes an approximate version of the language equivalence problem. To the best of our knowledge, the decision problem is still open, and has only be investigated for the special case when all states are connected~\cite{Tzeng92}.

\hide{
Since the scheduler attaining the maximally probable
behavior from a state may be different from the scheduler attaining
the maximally probable behavior from its neighbors, two
neighbors may still be distinguished by a scheduler. The
weaker definition does not preserve differential privacy. We hence
prefer the original definition.
}
