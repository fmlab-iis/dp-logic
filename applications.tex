


Consider the MDP in Figure~\ref{figure:simple-mdp-mdp}. Suppose the
states $+$ and $-$ are neighbors. Let us compute $\{ s : M,
\neighbor{S}, s \models \dpriv{3}{0} (\F Y) \}$. First, the self
product $M^2$ is depicted in Figure~\ref{figure:simple-mdp-product}
(the states $++$ and $--$ and associated transitions are ignored
for clarity). Applying
Theorem~\ref{theorem:eventuallyC} with $\epsilon = 4$ and $\delta = 0$, 
consider the following linear programming problem:
\[
\begin{array}{rcrcrcrcrcl}
  \multicolumn{11}{c}{
  \min x_{++} + x_{+-} + x_{-+} + x_{--} + x_{YY} + x_{YN} + x_{NY} + x_{NN}
  }\\
  \multicolumn{9}{r}{ x_{YN} } & = & 1 \\
  \multicolumn{9}{r}{ x_{YY} } & = & -3 \\
  \multicolumn{9}{r}{ x_{NY} } & = & -4 \\
  \multicolumn{9}{r}{ x_{NN} } & = & 0 \\
  x_{++} &-&   .5625 x_{YY} &-& .1875 x_{YN} &-& .1875 x_{NY} &-&
               .0625 x_{NN} & \geq & 0\\
  x_{++} &-&   .64   x_{YY} &-& .16   x_{YN} &-& .16   x_{NY} &-&
               .04   x_{NN} & \geq & 0\\
  x_{+-} &-&   .1875 x_{YY} &-& .5625 x_{YN} &-& .0625 x_{NY} &-& 
               .1875 x_{NN} & \geq & 0\\
  x_{+-} &-&   .16   x_{YY} &-& .64   x_{YN} &-& .04   x_{NY} &-&
               .16   x_{NN} & \geq & 0\\
  x_{-+} &-&   .1875 x_{YY} &-& .0625 x_{YN} &-& .5625 x_{NY} &-& 
               .1875 x_{NN} & \geq & 0\\
  x_{-+} &-&   .16   x_{YY} &-& .04   x_{YN} &-& .64   x_{NY} &-&
               .16   x_{NN} & \geq & 0\\
  x_{--} &-&   .0625 x_{YY} &-& .1875 x_{YN} &-& .1875 x_{NY} &-&
               .5625 x_{NN} & \geq & 0\\
  x_{--} &-&   .04   x_{YY} &-& .16   x_{YN} &-& .16   x_{NY} &-&
               .64   x_{NN} & \geq & 0
\end{array}
\]
The optimum is attained at 
$(x_{++}, x_{+-}, x_{-+}, x_{--}) = (-2.25, 0, -2.75, -0.6)$. By
Theorem~\ref{theorem:eventuallyC}, we have
\[
\begin{array}{l}
\max_{\scheduler{S}}
\myPr{+}{M_{\scheduler{S}}}{\neighbor{S}}{\F Y} -
4 \myPr{+}{M_{\scheduler{S}}}{\neighbor{S}}{\F Y}
= -2.25 \leq 0\\
\max_{\scheduler{S}}
\myPr{+}{M_{\scheduler{S}}}{\neighbor{S}}{\F Y} -
4 \myPr{-}{M_{\scheduler{S}}}{\neighbor{S}}{\F Y}
= 0 \leq 0 \\
\max_{\scheduler{S}}
\myPr{-}{M_{\scheduler{S}}}{\neighbor{S}}{\F Y} -
4 \myPr{+}{M_{\scheduler{S}}}{\neighbor{S}}{\F Y}
= -2.75 \leq 0\\
\max_{\scheduler{S}}
\myPr{-}{M_{\scheduler{S}}}{\neighbor{S}}{\F Y} -
4 \myPr{-}{M_{\scheduler{S}}}{\neighbor{S}}{\F Y}
= -0.6 \leq 0
\end{array}
\]
Therefore $+, - \in \{ s : M, \neighbor{M}, s \models \dpriv{3}{0}
(\F Y) \}$. This is expected. Recall that low- and high-accuracy
mechanisms are $(3, 0)$- and $(4, 0)$-differentially private
respectively. Regardless of request types, the interactive mechanism
in Figure~\ref{figure:simple-mdp-mdp} can only achieve $(4,
0)$-differential privacy at best.

In stream
processing mechanisms, input data are retrieved and analyzed
continuously. 

Continual observation is commonly found in on-line service
providers~\cite{DNPR:10:DPCO,DNPRY:10:PPSA}. 
Consider monitoring events $a$ and $b$ by streaming inputs. We would
like to compute the number of event types occurred in inputs. 

For each event, a bit denotes whether it has occurred or not. To model
adjacent inputs, we use four actions symbols to denote events and its
replacement. For instance, the action $a[a]$ means the event $a$ is
observed without replacement; $b[a]$ means the event $b$ is observed
but replaced with $a$ in an adjacent input. 

\begin{figure}
  \centering
  \begin{subfigure}{\columnwidth}
    \centering
      \begin{tikzpicture}[->,>=stealth',shorten >=1pt,auto,node
      distance=2cm,node/.style={circle,draw}]
      \node[node] (I) at  ( 0,  0) { $I$ };
      \node[node] (00) at (-1.5,  1.5) { $00$ };
      \node[node] (01) at (-1.5, -1.5) { $01$ };
      \node[node] (10) at ( 1.5,  1.5) { $10$ };
      \node[node] (11) at ( 1.5, -1.5) { $11$ };

      \path
      (I) edge node [below=2] { $1/4$ } (00)
      (I) edge node [above=2] { $1/4$ } (01)
      (I) edge node [below=2] { $1/4$ } (10)
      (I) edge node [above=2] { $1/4$ } (11)

      (00) edge [loop above] node { $a[\uscore]$, .375 } (00)
      (00) edge [bend left=10] node { $a[\uscore]$, .625 } (10)
      (00) edge [loop left] node [above=1,rotate=90] { $b[\uscore]$, .375 } (00)
      (00) edge [bend left=10] node [left=3,above=1,rotate=270] { $b[\uscore]$, .625 } (01)

      (10) edge [loop above] node { $a[\uscore]$, .625 } (00)
      (10) edge [bend left=10] node { $a[\uscore]$, .375 } (00)
      (10) edge [loop right] node [above=1,rotate=270] { $b[\uscore]$, .375 } (10)
      (10) edge [bend left=10] node [above=1,rotate=270] { $b[\uscore]$, .625 } (11)
     
      (11) edge [loop below] node { $a[\uscore]$, .625 } (11)
      (11) edge [bend left=10] node { $a[\uscore]$, .375 } (01)
      (11) edge [loop right] node [above=1,rotate=270] { $b[\uscore]$, .625 } (11)
      (11) edge [bend left=10] node [left=-3,above=1,rotate=90] { $b[\uscore]$, .375 } (10)
     
      (01) edge [loop below] node { $a[\uscore]$, .375 } (01)
      (01) edge [bend left=10] node { $a[\uscore]$, .675 } (11)
      (01) edge [loop left] node [above=1,rotate=90] { $b[\uscore]$, .625 } (01)
      (01) edge [bend left=10] node [left=-3,above=1,rotate=90] { $b[\uscore]$, .375 } (00)
     
      ;



      \node[node] (I') at  ( 5.5,    0) { $\underline{I}$ };
      \node[node] (00') at (   4,  1.5) { $\underline{00}$ };
      \node[node] (01') at (   4, -1.5) { $\underline{01}$ };
      \node[node] (10') at (   7,  1.5) { $\underline{10}$ };
      \node[node] (11') at (   7, -1.5) { $\underline{11}$ };

      \path
      (I') edge node [below=2] { $1/4$ } (00')
      (I') edge node [above=2] { $1/4$ } (01')
      (I') edge node [below=2] { $1/4$ } (10')
      (I') edge node [above=2] { $1/4$ } (11')

      (00') edge [loop above] node { $\uscore[a]$, .375 } (00')
      (00') edge [bend left=10] node { $\uscore[a]$, .625 } (10')
      (00') edge [loop left] node [above=1,rotate=90] { $\uscore[b]$, .375 } (00')
      (00') edge [bend left=10] node [left=3,above=1,rotate=270] { $\uscore[b]$, .625 } (01')

      (10') edge [loop above] node { $\uscore[a]$, .625 } (00')
      (10') edge [bend left=10] node { $\uscore[a]$, .375 } (00')
      (10') edge [loop right] node [above=1,rotate=270] { $\uscore[b]$, .375 } (10')
      (10') edge [bend left=10] node [above=1,rotate=270] { $\uscore[b]$, .625 } (11')
     
      (11') edge [loop below] node { $\uscore[a]$, .625 } (11')
      (11') edge [bend left=10] node { $\uscore[a]$, .375 } (01')
      (11') edge [loop right] node [above=1,rotate=270] { $\uscore[b]$, .625 } (11')
      (11') edge [bend left=10] node [left=-3,above=1,rotate=90] { $\uscore[b]$, .375 } (10')
     
      (01') edge [loop below] node { $\uscore[a]$, .375 } (01')
      (01') edge [bend left=10] node { $\uscore[a]$, .675 } (11')
      (01') edge [loop left] node [above=1,rotate=90] { $\uscore[b]$, .625 } (01')
      (01') edge [bend left=10] node [left=-3,above=1,rotate=90] { $\uscore[b]$, .375 } (00')
     
      ;
      \end{tikzpicture}
      \caption{Adjacent Inputs}
      \label{figure:adjacent-inputs}
    \end{subfigure}
    \begin{subfigure}{\columnwidth}
      \centering
      \begin{tikzpicture}[->,>=stealth',shorten >=1pt,auto,node
      distance=2cm,node/.style={circle,draw}]
      \node[node] (0) at (-5,  0) { $0$ };
      \node[node] (1) at ( 0,  0) { $1$ };
      \node[node] (2) at ( 5,  0) { $2$ };

      \node[node] (00) at (-4.5, 1.5) { $00$ };
      \node[node] (01) at (-1.5, 2) { $01$ };
      \node[node] (10) at ( 1.5, 2) { $10$ };
      \node[node] (11) at ( 4.5, 1.5) { $11$ };

      \node[node] (00') at (-4.5, -1.5) { $\underline{00}$ };
      \node[node] (01') at (-1.5, -2) { $\underline{01}$ };
      \node[node] (10') at ( 1.5, -2) { $\underline{10}$ };
      \node[node] (11') at ( 4.5, -1.5) { $\underline{11}$ };

      \path
      (00) edge node [left] { $2/3$ } (0)
      (00) edge node [left=50,above] { $1/6$ } (1)
      (00) edge node [left=100,above=15] { $1/6$ } (2)

      (01) edge node [right=30,above=20] { $1/3$ } (0)
      (01) edge node [left=20,above=5] { $1/3$ } (1)
      (01) edge node [left=70,above=20] { $1/3$ } (2)

      (10) edge node [right=70,above=20] { $1/3$ } (0)
      (10) edge node [right=20,above=5] { $1/3$ } (1)
      (10) edge node [left=30,above=20] { $1/3$ } (2)

      (11) edge node [right=100,above=15] { $1/6$ } (0)
      (11) edge node [right=50,above] { $1/6$ } (1)
      (11) edge node [right] { $2/3$ } (2)


      (00') edge node [left] { $2/3$ } (0)
      (00') edge node [left=50,below] { $1/6$ } (1)
      (00') edge node [left=100,below=15] { $1/6$ } (2)

      (01') edge node [right=30,below=20] { $1/3$ } (0)
      (01') edge node [left=20,below=5] { $1/3$ } (1)
      (01') edge node [left=70,below=20] { $1/3$ } (2)

      (10') edge node [right=70,below=20] { $1/3$ } (0)
      (10') edge node [right=20,below=5] { $1/3$ } (1)
      (10') edge node [left=30,below=20] { $1/3$ } (2)

      (11') edge node [right=100,below=15] { $1/6$ } (0)
      (11') edge node [right=50,below] { $1/6$ } (1)
      (11') edge node [right] { $2/3$ } (2)
      ;

      \end{tikzpicture}
      \caption{Randomized Frequency}
      \label{figure:randomized-frequency}
    \end{subfigure}
  \caption{Frequency Estimator}
  \label{figure:frequency-estimator}
\end{figure}

\todo{verify this is $1$-DP}