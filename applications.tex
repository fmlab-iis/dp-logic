
Streaming processing gives a motivating example for Markov
decision processes. In stream processing, input data are retrieved and
analyzed continuously. Such continual observation is commonly found
in social media, service providers, or even electronic surveillance
systems. To protect privacy, privacy-respecting mechanisms must be
formulated and designed. Classical differential privacy however is
insufficient in streaming processing. Note that observations are of
indefinite lengths in such applications. All differentially private
mechanism necessarily stay in their internal states indefinitely
before outcomes are produced. Although differential privacy demands
privacy-respecting outcomes, internal states may reveal private
information in such mechanisms. When differentially private mechanisms
are in their internal states, privacy is still lost if an intrusion
occurs.

In pan privacy, data analysis mechanisms are required to respect
privacy even in their internal states~\cite{DNPR:10:DPCO,DNPRY:10:PPSA}.
The idea is to show reachability probabilities are similar on
neighboring observations. In the following, we model a discrete
variant of the mechanisms by Markov decision processes. We will also
specify pan privacy in our model using $\dpCTL$.

Consider monitoring events of the types $a$ and $b$ continuously.
A sequence of event types is said to be an observation. Two
obsevations are {neighbors} if one can be obtained from the other
by replacing event types any number of times, that is, if they are of
the same length. %\bw{Later, we will consider a different definition of
%neighborhood by restricting the number of replacements.}
We would like to inquire the number of event types in
observations. For instance, the query result is $1$
for observations $a \cdots a$ and $b \cdots b$ of arbitrary positive
lengths; and it is $2$ for other non-empty observations.

Based on the density estimation algorithm in \cite{DNPRY:10:PPSA}, we
consider the following mechanism for event type streams. Our mechanism
keeps a bit for each event type. The bits are initialized to $0$ or
$1$ uniformly. On observing an event type in the input stream, the
bit for the event type is set to $0$ and $1$ with probability
$\frac{3}{8}$ and $\frac{5}{8}$ respectively. The mechanism continues
to update its bits until it is asked to return the number of observed
event types. To compute the number of observed event types, the
mechanism adopts the truncated $1/2$-geometric mechanism on the sum of
two bits for event types. If the sum is $0$, the mechanism outputs $0$
with probability $\frac{2}{3}$ and $1, 2$ each with probability
$\frac{1}{6}$. If the sum is $1$, one of $0, 1, 2$ is returned with
the uniform distribution. If the sum is $2$, the mechanism outputs $2$
with probability $\frac{2}{3}$ and $0, 1$ each with probability
$\frac{1}{6}$ (Algorithm~\ref{algorithm:frequency-estimator}).

\begin{algorithm}
  \begin{algorithmic}[1]
  \Procedure{FrequencyEstimator}{}
    \State{$f_a \leftarrow 0, 1$ with probability
      $\frac{1}{2}, \frac{1}{2}$ respectively}
    \Comment{Initialization}
    \State{$f_b \leftarrow 0, 1$ with probability
      $\frac{1}{2}, \frac{1}{2}$ respectively}
    \Match{current event type $e$}
    \Comment{Processing}
        \lCase{$a$}{$f_a \leftarrow 0, 1$ with probability
            $\frac{3}{8}, \frac{5}{8}$ respectively}
        \lCase{$b$}{$f_b \leftarrow 0, 1$ with probability
            $\frac{3}{8}, \frac{5}{8}$ respectively}
    \EndMatch
    \Match{$f_a + f_b$}
    \Comment{Output}
      \lCase{0}{$\mathit{out} \leftarrow 0, 1, 2$ with probability
          $\frac{2}{3}, \frac{1}{6}, \frac{1}{6}$ respectively}
      \lCase{1}{$\mathit{out} \leftarrow 0, 1, 2$ with probability
          $\frac{1}{3}, \frac{1}{3}, \frac{1}{3}$ respectively}
      \lCase{2}{$\mathit{out} \leftarrow 0, 1, 2$ with probability
          $\frac{1}{6}, \frac{1}{6}, \frac{2}{3}$ respectively}
    \EndMatch
    \State{\Return $\mathit{out}$}
  \EndProcedure
  \end{algorithmic}
  \caption{FrequencyEstimator}
  \label{algorithm:frequency-estimator}
\end{algorithm}

\indent
\emph{Privacy analysis}.
In the processing phase, the bit $f_a$ for an event type $a$ can only
have two different distributions $U$ and $B$. If the event type is
never observed, $\Pr_U[f_a = 0] = \Pr_U[f_a = 1] = \frac{1}{2}$; if
the event type has been observed, we have $\Pr_B[f_a = 0] =
\frac{3}{8}$ and $\Pr_B[f_a = 1] = \frac{5}{8}$. Note that
$\frac{1}{2} \cdot \frac{3}{8} \leq \frac{1}{2} \leq 2 \cdot
\frac{3}{8}$ and $\frac{1}{2} \cdot \frac{5}{8} \leq \frac{1}{2} \leq
2 \cdot \frac{5}{8}$. For every $x \in \{ 0, 1 \}$, we have
$\frac{1}{2} \cdot \Pr_B[f_a = x] \leq \Pr_U[f_a = x] \leq 2 \cdot
\Pr_B[f_a = x]$. On neighboring observations, the distribution of
$f_a$ can only be changed from $U$ to $B$, values of $f_a$ is $(2,
0)$-differentially private and similarly for $f_b$. Finally, the
$1/2$-geometric mechanism gives $(\ln 2, 0)$-differential privacy.
Recall $2\ln 2 = \ln 4$,
the steaming data analysis mechanism in
Algorithm~\ref{algorithm:frequency-estimator} is $(\ln 4,
0)$-differentially private by
composition theorem.

To model Algorithm~\ref{algorithm:frequency-estimator}, let
us start with the formalization of observations. Note that the
lengths of observations are finite but unbounded. An infinite Markov
chain would be necessary if observations were formalized as states.
Recall that neighbors are observations of the same length. If we
furthermore consider the formalization of neighbors, it is not clear
how to construct a finite Markov chain for
Algorithm~\ref{algorithm:frequency-estimator}.

In order to obtain a finite model, we formalize neighbors as action
sequences in Markov decision processes. Consider the action set $\{
a[a], a[b], b[a], b[b] \}$. Informally, the action $x[y]$ means the
event type $x$ is replaced by the event type $y$ in a neighbor. An
action sequence thus represents an observation and one of its
neighbors and vice versa. More formally, the action sequence
$x_1[y_1]x_2[y_2]\cdots x_n[y_n]$ represents the observation
$x_1x_2\cdots x_n$ and its neighbor $y_1y_2\cdots y_n$. For instance,
the action sequence $a[a]a[b]a[a]b[a]b[b]$ represents the observation
$aaabb$ and a neighbor $abaab$.

Similar to Algorithm~\ref{algorithm:frequency-estimator}, our Markov
decision process has internal states $00$, $01$, $10$, $11$ for the
two bits $f_a$ and $f_b$ of observed event types
(Figure~\ref{figure:adjacent-inputs}). The internal state $01$
represents $(f_a, f_b) = (0, 1)$, for instance. From the
initialization state $I$, the Markov decision process moves to one of
the four internal states randomly with the uniform
distribution. Consider the internal state $01$. If the actions $a[a]$
or $a[b]$ are chosen, they denote that the event type $a$ is currently
observed. The Markov decision process moves to the internal state $01$
and $11$ with probability $.375 = \frac{3}{8}$ and $.625 =
\frac{5}{8}$ respectively. Other transitions are defined similarly.

On the other hand, the initialization state $\underline{I}$
corresponds to the computation on neighbors. The Markov decision
process moves to the interal states $\underline{00}$,
$\underline{01}$, $\underline{10}$, $\underline{11}$ randomly with
the uniform distribution. Consider, for example, the internal state
$\underline{10}$. If the action $a[a]$ or $b[a]$ are chosen, they
denote that the current event type is replaced with the eveny type
$a$. The Markov decision process moves to the internal state
$\underline{00}$ and $\underline{10}$ with probability $.375 =
\frac{3}{8}$ and $.625 = \frac{5}{8}$ respectively. At an internal
state, the Markov decision process can move to the output states $0$,
$1$, or $2$ by the truncated $\frac{1}{2}$-geometric mechanism over
$\underline{2}$ (Figure~\ref{figure:randomized-frequency}, \`{a} la
Figure~\ref{figure:geometric-mechanism}). For any action sequence,
observe that paths from the state $I$ and those from $\underline{I}$
formalize the computation of Algorithm~\ref{algorithm:frequency-estimator}
on an observation and its neighbor represented by the sequence. We
hence say the states $I$ and $\underline{I}$ are neighbors in the
Markov decision process (Figure~\ref{figure:frequency-estimator}).

\begin{figure}
  \centering
  \begin{subfigure}{\columnwidth}
    \centering
      \begin{tikzpicture}[->,>=stealth',shorten >=1pt,auto,node
      distance=2cm,node/.style={circle,draw}]
      \node[node] (I) at  ( 0,  0) { $I$ };
      \node[node] (00) at (-1.5,  1.5) { $00$ };
      \node[node] (01) at (-1.5, -1.5) { $01$ };
      \node[node] (10) at ( 1.5,  1.5) { $10$ };
      \node[node] (11) at ( 1.5, -1.5) { $11$ };

      \path
      (I) edge node [below=2] { $1/4$ } (00)
      (I) edge node [above=2] { $1/4$ } (01)
      (I) edge node [below=2] { $1/4$ } (10)
      (I) edge node [above=2] { $1/4$ } (11)

      (00) edge [loop above] node { $a[\uscore]$, .375 } (00)
      (00) edge [bend left=10] node { $a[\uscore]$, .625 } (10)
      (00) edge [loop left] node [above=1,rotate=90] { $b[\uscore]$, .375 } (00)
      (00) edge [bend left=10] node [left=3,above=1,rotate=270] { $b[\uscore]$, .625 } (01)

      (10) edge [loop above] node { $a[\uscore]$, .625 } (00)
      (10) edge [bend left=10] node { $a[\uscore]$, .375 } (00)
      (10) edge [loop right] node [above=1,rotate=270] { $b[\uscore]$, .375 } (10)
      (10) edge [bend left=10] node [above=1,rotate=270] { $b[\uscore]$, .625 } (11)

      (11) edge [loop below] node { $a[\uscore]$, .625 } (11)
      (11) edge [bend left=10] node { $a[\uscore]$, .375 } (01)
      (11) edge [loop right] node [above=1,rotate=270] { $b[\uscore]$, .625 } (11)
      (11) edge [bend left=10] node [left=-3,above=1,rotate=90] { $b[\uscore]$, .375 } (10)

      (01) edge [loop below] node { $a[\uscore]$, .375 } (01)
      (01) edge [bend left=10] node { $a[\uscore]$, .625 } (11)
      (01) edge [loop left] node [above=1,rotate=90] { $b[\uscore]$, .625 } (01)
      (01) edge [bend left=10] node [left=-3,above=1,rotate=90] { $b[\uscore]$, .375 } (00)

      ;



      \node[node] (I') at  ( 5.5,    0) { $\underline{I}$ };
      \node[node] (00') at (   4,  1.5) { $\underline{00}$ };
      \node[node] (01') at (   4, -1.5) { $\underline{01}$ };
      \node[node] (10') at (   7,  1.5) { $\underline{10}$ };
      \node[node] (11') at (   7, -1.5) { $\underline{11}$ };

      \path
      (I') edge node [below=2] { $1/4$ } (00')
      (I') edge node [above=2] { $1/4$ } (01')
      (I') edge node [below=2] { $1/4$ } (10')
      (I') edge node [above=2] { $1/4$ } (11')

      (00') edge [loop above] node { $\uscore[a]$, .375 } (00')
      (00') edge [bend left=10] node { $\uscore[a]$, .625 } (10')
      (00') edge [loop left] node [above=1,rotate=90] { $\uscore[b]$, .375 } (00')
      (00') edge [bend left=10] node [left=3,above=1,rotate=270] { $\uscore[b]$, .625 } (01')

      (10') edge [loop above] node { $\uscore[a]$, .625 } (00')
      (10') edge [bend left=10] node { $\uscore[a]$, .375 } (00')
      (10') edge [loop right] node [above=1,rotate=270] { $\uscore[b]$, .375 } (10')
      (10') edge [bend left=10] node [above=1,rotate=270] { $\uscore[b]$, .625 } (11')

      (11') edge [loop below] node { $\uscore[a]$, .625 } (11')
      (11') edge [bend left=10] node { $\uscore[a]$, .375 } (01')
      (11') edge [loop right] node [above=1,rotate=270] { $\uscore[b]$, .625 } (11')
      (11') edge [bend left=10] node [left=-3,above=1,rotate=90] { $\uscore[b]$, .375 } (10')

      (01') edge [loop below] node { $\uscore[a]$, .375 } (01')
      (01') edge [bend left=10] node { $\uscore[a]$, .625 } (11')
      (01') edge [loop left] node [above=1,rotate=90] { $\uscore[b]$, .625 } (01')
      (01') edge [bend left=10] node [left=-3,above=1,rotate=90] { $\uscore[b]$, .375 } (00')

      ;
      \end{tikzpicture}
      \caption{Adjacent Inputs}
      \label{figure:adjacent-inputs}
    \end{subfigure}
    \begin{subfigure}{\columnwidth}
      \centering
      \begin{tikzpicture}[->,>=stealth',shorten >=1pt,auto,node
      distance=2cm,node/.style={circle,draw}]
      \node[node] (0) at (-5,  0) { $0$ };
      \node[node] (1) at ( 0,  0) { $1$ };
      \node[node] (2) at ( 5,  0) { $2$ };

      \node[node] (00) at (-4.5, 1.5) { $00$ };
      \node[node] (01) at (-1.5, 2) { $01$ };
      \node[node] (10) at ( 1.5, 2) { $10$ };
      \node[node] (11) at ( 4.5, 1.5) { $11$ };

      \node[node] (00') at (-4.5, -1.5) { $\underline{00}$ };
      \node[node] (01') at (-1.5, -2) { $\underline{01}$ };
      \node[node] (10') at ( 1.5, -2) { $\underline{10}$ };
      \node[node] (11') at ( 4.5, -1.5) { $\underline{11}$ };

      \path
      (00) edge node [left] { $2/3$ } (0)
      (00) edge node [left=50,above] { $1/6$ } (1)
      (00) edge node [left=100,above=15] { $1/6$ } (2)

      (01) edge node [right=30,above=20] { $1/3$ } (0)
      (01) edge node [left=20,above=5] { $1/3$ } (1)
      (01) edge node [left=70,above=20] { $1/3$ } (2)

      (10) edge node [right=70,above=20] { $1/3$ } (0)
      (10) edge node [right=20,above=5] { $1/3$ } (1)
      (10) edge node [left=30,above=20] { $1/3$ } (2)

      (11) edge node [right=100,above=15] { $1/6$ } (0)
      (11) edge node [right=50,above] { $1/6$ } (1)
      (11) edge node [right] { $2/3$ } (2)


      (00') edge node [left] { $2/3$ } (0)
      (00') edge node [left=50,below] { $1/6$ } (1)
      (00') edge node [left=100,below=15] { $1/6$ } (2)

      (01') edge node [right=30,below=20] { $1/3$ } (0)
      (01') edge node [left=20,below=5] { $1/3$ } (1)
      (01') edge node [left=70,below=20] { $1/3$ } (2)

      (10') edge node [right=70,below=20] { $1/3$ } (0)
      (10') edge node [right=20,below=5] { $1/3$ } (1)
      (10') edge node [left=30,below=20] { $1/3$ } (2)

      (11') edge node [right=100,below=15] { $1/6$ } (0)
      (11') edge node [right=50,below] { $1/6$ } (1)
      (11') edge node [right] { $2/3$ } (2)
      ;

      \end{tikzpicture}
      \caption{Randomized Frequency}
      \label{figure:randomized-frequency}
    \end{subfigure}
  \caption{Frequency Estimator}
  \label{figure:frequency-estimator}
\end{figure}

\noindent
\emph{Privacy Specification}.
For classical differential privacy, we are interested in the output
distribution. Algorithm~\ref{algorithm:frequency-estimator} has three
different outputs. By privacy analysis, the state $I$ satisfies the
following $\dpCTLstar$ formula:
\[
\dpriv{\ln 4}{0} \F 0 \wedge \dpriv{\ln 4}{0} \F 1 \wedge
\dpriv{\ln 4}{0} \F 2
\]
For pan privacy, we moreover wish to measure privacy loss in internal
states. For observations of length $k$, the probabilities of reaching
each internal states need to be sufficiently close. The state $I$
therefore must satisfy
\[
\dpriv{\ln 4}{0} \F 0 \wedge \dpriv{\ln 4}{0} \F 1 \wedge
\dpriv{\ln 4}{0} \F 2
\wedge
\bigwedge_{k \in \bbfN}
\dpriv{\ln 4}{0} \X^k 00 \wedge
\dpriv{\ln 4}{0} \X^k 01 \wedge
\dpriv{\ln 4}{0} \X^k 10 \wedge
\dpriv{\ln 4}{0} \X^k 11.
\]
