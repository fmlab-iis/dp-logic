
Below we describe an online mechanism from~\cite{DR:14:AFDP}.
Given a threshold and a series of adaptive queries, we care for
the queries whose results are above the threshold; queries below the
threshold only disclose minimal information and hence are considered
irrelevant. Since query results above the threshold convey useful
information, let us assume the mechanism will halt on the first such
query result for simplicity. 
%The question is to design a privacy-preserving mechanism to have 
%similar behaviors on similar databases. 
In~\cite{DR:14:AFDP}, a mechanism is design for continuous
queries by applying the Laplace mechanism. We will develop a mechanism
for bounded and discrete queries using the geometric mechanism.

\begin{algorithm}
	\begin{algorithmic}[1]
		\Procedure{AboveThreshold}{$d$, $\{ f_1, f_2, \ldots
                  \}$, $t$}
		\Match{$t$}
		\Comment{obtain $t'$ by $\frac{1}{4}$-geometric mechanism}
		\lCase{$0$}{$t' \leftarrow 0, 1,2,3,4,5$ with probability
			$\frac{4}{5}$,$\frac{3}{20}$,
			$\frac{3}{80}$,$\frac{3}{320}$,
			$\frac{3}{1280}$,$\frac{1}{1280}$ respectively}
		\lCase{$1$}{$t' \leftarrow 0, 1,2,3,4,5$ with probability
			$\frac{1}{5}$,$\frac{3}{5}$,
			$\frac{3}{20}$,$\frac{3}{80}$,
			$\frac{3}{320}$,$\frac{1}{320}$ respectively}
		\lCase{$2$}{$t' \leftarrow 0, 1,2,3,4,5$ with probability
			$\frac{1}{20}$,$\frac{3}{20}$,
			$\frac{3}{5}$,$\frac{3}{20}$,
			$\frac{3}{80}$,$\frac{1}{80}$ respectively}
		\lCase{$3$}{$t' \leftarrow 0, 1,2,3,4,5$ with probability
			$\frac{1}{80}$,$\frac{3}{80}$,
			$\frac{3}{20}$,$\frac{3}{5}$,
			$\frac{3}{20}$,$\frac{1}{20}$ respectively}
		\lCase{$4$}{$t' \leftarrow 0, 1,2,3,4,5$ with probability
			$\frac{1}{320}$,$\frac{3}{320}$,
			$\frac{3}{80}$,$\frac{3}{20}$,
			$\frac{3}{5}$,$\frac{1}{5}$ respectively}
		\lCase{$5$}{$t' \leftarrow 0, 1,2,3,4,5$ with probability
			$\frac{1}{1280}$,$\frac{3}{1280}$,
			$\frac{3}{320}$,$\frac{3}{80}$,
			$\frac{3}{20}$,$\frac{4}{5}$ respectively}
		\EndMatch
		\For{each query $f_i$}
                \State{$r_i \leftarrow f_i (d)$}
		\Match{ $r_i$}
		\Comment{obtain $r_{i}'$ by $\frac{1}{2}$-geometric mechanism}
		\lCase{$0$}{$r_{i}' \leftarrow 0, 1,2,3,4,5$ with probability
			$\frac{2}{3}$,$\frac{1}{6}$,
			$\frac{1}{12}$,$\frac{1}{24}$,
			$\frac{1}{48}$,$\frac{1}{48}$ respectively}
		\lCase{$1$}{$r_{i}' \leftarrow 0, 1,2,3,4,5$ with probability
			$\frac{1}{3}$,$\frac{1}{3}$,
			$\frac{1}{6}$,$\frac{1}{12}$,
			$\frac{1}{24}$,$\frac{1}{24}$ respectively}
		\lCase{$2$}{$r_{i}' \leftarrow 0, 1,2,3,4,5$ with probability
			$\frac{1}{6}$,$\frac{1}{6}$,
			$\frac{1}{3}$,$\frac{1}{6}$,
			$\frac{1}{12}$,$\frac{1}{12}$ respectively}
		\lCase{$3$}{$r_{i}' \leftarrow 0, 1,2,3,4,5$ with probability
			$\frac{1}{12}$,$\frac{1}{12}$,
			$\frac{1}{6}$,$\frac{1}{3}$,
			$\frac{1}{6}$,$\frac{1}{6}$ respectively}
		\lCase{$4$}{$r_{i}' \leftarrow 0, 1,2,3,4,5$ with probability
			$\frac{1}{24}$,$\frac{1}{24}$,
			$\frac{1}{12}$,$\frac{1}{6}$,
			$\frac{1}{3}$,$\frac{1}{3}$ respectively}
		\lCase{$5$}{$r_{i}' \leftarrow 0, 1,2,3,4,5$ with probability
			$\frac{1}{48}$,$\frac{1}{48}$,
			$\frac{1}{24}$,$\frac{1}{12}$,
			$\frac{1}{6}$,$\frac{2}{3}$ respectively}
		\EndMatch
                \State{\textbf{if} $r'_i \geq t'$ \textbf{then} 
                  \textbf{halt} with $a_i = \top$ \textbf{else}
                  $a_i = \bot$}
\hide{
		\If{$r_{i}'>t'$}
		\State{$a_i=\top$}
		\State{\textbf{halt}}
		\Else
		\State{$a_i=\bot$}
		\EndIf
}
		\EndFor
		\EndProcedure
		
	\end{algorithmic}
	\caption{Input: private database $d$, 
          queries $f_i : \{ 0, 1, \ldots, 5 \} \rightarrow \{ 0, 1,
          \ldots, 5 \}$ with sensitivity $1$, threshold $t \in \{ 0,
          1, \ldots, 5 \}$; Output: $a_1, a_2, \ldots$}
	\label{algorithm:online-model}
\end{algorithm}

Consider a threshold $t \in
\{ 0, 1 \ldots, 5\}$, and queries $\{ f_i : f_i $ with sensitivity $\Delta
(f_i) = 1 \}$. In order to protect privacy, our mechanism applies the
truncated $\frac{1}{4}$-geometric mechanism to obtain a perturbed
threshold $t'$. For each query, the truncated $\frac{1}{2}$-geometric
mechanism is applied to its result. If the perturbed result is not less
than the perturbed threshold, our mechanism outputs $\top$ and
halts. Otherwise, it outputs $\bot$ and continues to the next query
(Algorithm~\ref{algorithm:online-model}). The above threshold
mechanism outputs a sequence of the form $\bot^* \top$. On
similar databases, we would like to show that the above
threshold mechanism outputs the same sequence with similar
probabilities. 

\begin{figure}
  \centering
    \resizebox{.8\columnwidth}{!}{
    \begin{tikzpicture}[->,>=stealth',shorten >=1pt,auto,node
      distance=2cm,node/.style={circle,draw}]
      \node[node] (t0r1) at (0, 4.25) { $t_0r_1$ };

      \draw (2.9, 3.45) rectangle (4.9, .05);
      \node at (4.65, 2.75) { $\cdots $};
      \node[node] (t'0r1) at (3.9, 2.75) { $t'_0r_1$ };
      \node at (3.2, 2.75) { $\cdots $};
      \draw (-2.8, 3.45) rectangle (2.8, .05);
      \node[node] (t'1r0) at (2.1, 2.75) { $t'_1r_0$ };
      \node[node] (t'1r1) at (0, 2.75) { $t'_1r_1$ };
      \node[node] (t'1r2) at (-2.1, 2.75) { $t'_1r_2$ };
      \node at (0, 1.85) { $\vdots$ };
      \node at (-2.1, 1.85) { $\vdots$ };
      \draw (-2.9, 3.45) rectangle (-4.9, .05);
      \node at (-3.15, 2.75) { $\cdots$ };
      \node[node] (t'2r1) at (-3.9, 2.75) { $t'_2r_1$ };
      \node at (-4.6, 2.75) { $\cdots$ };

      \node[node] (t'1r'0) at (2.1, .75) { $t'_1r'_0$ };
      \node[node] (t'1r'1) at (0, .75) { $t'_1r'_1$ };
      \node[node] (t'1r'2) at (-2.1, .75) { $t'_1r'_2$ };

      \node[node] (tt0rr0) at (2.1, -4.25) { $\underline{t}_0\underline{r}_0$ };

      \draw (2.9, -3.45) rectangle (4.9, -.05);
      \node[node] (tt'0rr0) at (4.2, -2.75) { $\underline{t}'_0\underline{r}_0$ };
      \node at (3.35, -2.75) { $\cdots$ };
      \draw (.8, -3.45) rectangle (2.8, -.05);
      \node[node] (tt'1rr0) at (2.1, -2.75) { $\underline{t}'_1\underline{r}_0$ };
      \node at (1.25, -2.75) { $\cdots$ };

      \draw (.7, -3.45) rectangle (-4.9, -.05);
      \node[node] (tt'2rr0) at (0, -2.75) { $\underline{t}'_2\underline{r}_0$ };
      \node[node] (tt'2rr1) at (-2.1, -2.75) { $\underline{t}'_2\underline{r}_1$ };
      \node[node] (tt'2rr2) at (-4.2, -2.75) { $\underline{t}'_2\underline{r}_2$ };

      \node at (0, -1.65) { $\vdots$ };
      
      \node[node] (tt'2rr'0) at (0, -.75) { $\underline{t}'_2\underline{r}'_0$ };
      \node[node] (tt'2rr'1) at (-2.1, -.75) { $\underline{t}'_2\underline{r}'_1$ };
      \node at (-4.2, -1.65) { $\vdots$ };

      \node[node] (tt'2rr'2) at (-4.2, -.75) { $\underline{t}'_2\underline{r}'_2$ };


      \path
      (t0r1) edge node [above] {  } (t'0r1)
      (t0r1) edge node [left] {  } (t'1r1)
      (t0r1) edge node [above] {  } (t'2r1)

%      (t'1r1) edge node [above] { $\frac{1}{1}$ } (t'1r'0)
%      (t'1r1) edge node [above] { $\frac{1}{1}$ } (t'1r'1)
%      (t'1r1) edge node [above] { $\frac{1}{1}$ } (t'1r'2)

      (t'1r'0) edge [double] node [above]
              { \hide{$0|0, 0|1$} } (t'1r0)
      (t'1r'0) edge [double] node [above] 
              { \hide{$1|0, 1|1, 1|2$} } (t'1r1)
      (t'1r'0) edge [double] node [below] 
              { \hide{$2|1, 2|2$} } (t'1r2)

      (tt0rr0) edge node [below] {  } (tt'0rr0)
      (tt0rr0) edge node [left] {  } (tt'1rr0)
      (tt0rr0) edge node [below] {  } (tt'2rr0)

%      (tt'2rr'0) edge [double] node [above] { $0|0, 1|0$ } (tt'2rr0)
%      (tt'2rr'0) edge [double] node [above] { $0|1, 1|1, 2|1$ } (tt'2rr1)
%      (tt'2rr'0) edge [double] node [above] { $1|2, 2|2$ } (tt'2rr2)

      (tt'2rr'1) edge [double] node [above]
                { \hide{$0|0, 1|0$} } (tt'2rr0)
      (tt'2rr'1) edge [double] node [above]
                { \hide{$0|1, 1|1, 2|1$} } (tt'2rr1)
      (tt'2rr'1) edge [double] node [above]
                { \hide{$1|2, 2|2$} } (tt'2rr2)

      ;
      \end{tikzpicture}
    }
  \caption{Markov Decision Process for Above Threshold}
  \label{figure:mdp-above-threshold}
\end{figure}

