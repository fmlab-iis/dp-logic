
Below we describe an online mechanism from~\cite{DR:14:AFDP}.
Given a threshold and a series of adaptive queries, we care for
the queries whose results are above the threshold; queries below the
threshold only disclose minimal information and hence is
irrelevant. Let us assume the mechanism will halt on the first such
query result for simplicity. 
%The question is to design a privacy-preserving mechanism to have 
%similar behaviors on similar databases. 
In~\cite{DR:14:AFDP}, a mechanism is designed for continuous
queries by applying the Laplace mechanism. We will develop a mechanism
for discrete bounded queries using the truncated geometric mechanism.

\begin{algorithm}
	\begin{algorithmic}[1]
		\Procedure{AboveThreshold}{$d$, $\{ f_1, f_2, \ldots
                  \}$, $t$}
		\Match{$t$}
		\Comment{obtain $t'$ by $\frac{1}{4}$-geometric mechanism}
		\lCase{$0$}{$t' \leftarrow 0, 1,2,3,4,5$ with probability
			$\frac{4}{5}$,$\frac{3}{20}$,
			$\frac{3}{80}$,$\frac{3}{320}$,
			$\frac{3}{1280}$,$\frac{1}{1280}$ respectively}
		\lCase{$1$}{$t' \leftarrow 0, 1,2,3,4,5$ with probability
			$\frac{1}{5}$,$\frac{3}{5}$,
			$\frac{3}{20}$,$\frac{3}{80}$,
			$\frac{3}{320}$,$\frac{1}{320}$ respectively}
		\lCase{$2$}{$t' \leftarrow 0, 1,2,3,4,5$ with probability
			$\frac{1}{20}$,$\frac{3}{20}$,
			$\frac{3}{5}$,$\frac{3}{20}$,
			$\frac{3}{80}$,$\frac{1}{80}$ respectively}
		\lCase{$3$}{$t' \leftarrow 0, 1,2,3,4,5$ with probability
			$\frac{1}{80}$,$\frac{3}{80}$,
			$\frac{3}{20}$,$\frac{3}{5}$,
			$\frac{3}{20}$,$\frac{1}{20}$ respectively}
		\lCase{$4$}{$t' \leftarrow 0, 1,2,3,4,5$ with probability
			$\frac{1}{320}$,$\frac{3}{320}$,
			$\frac{3}{80}$,$\frac{3}{20}$,
			$\frac{3}{5}$,$\frac{1}{5}$ respectively}
		\lCase{$5$}{$t' \leftarrow 0, 1,2,3,4,5$ with probability
			$\frac{1}{1280}$,$\frac{3}{1280}$,
			$\frac{3}{320}$,$\frac{3}{80}$,
			$\frac{3}{20}$,$\frac{4}{5}$ respectively}
		\EndMatch
		\For{each query $f_i$}
                \State{$r_i \leftarrow f_i (d)$}
		\Match{ $r_i$}
		\Comment{obtain $r_{i}'$ by $\frac{1}{2}$-geometric mechanism}
		\lCase{$0$}{$r_{i}' \leftarrow 0, 1,2,3,4,5$ with probability
			$\frac{2}{3}$,$\frac{1}{6}$,
			$\frac{1}{12}$,$\frac{1}{24}$,
			$\frac{1}{48}$,$\frac{1}{48}$ respectively}
		\lCase{$1$}{$r_{i}' \leftarrow 0, 1,2,3,4,5$ with probability
			$\frac{1}{3}$,$\frac{1}{3}$,
			$\frac{1}{6}$,$\frac{1}{12}$,
			$\frac{1}{24}$,$\frac{1}{24}$ respectively}
		\lCase{$2$}{$r_{i}' \leftarrow 0, 1,2,3,4,5$ with probability
			$\frac{1}{6}$,$\frac{1}{6}$,
			$\frac{1}{3}$,$\frac{1}{6}$,
			$\frac{1}{12}$,$\frac{1}{12}$ respectively}
		\lCase{$3$}{$r_{i}' \leftarrow 0, 1,2,3,4,5$ with probability
			$\frac{1}{12}$,$\frac{1}{12}$,
			$\frac{1}{6}$,$\frac{1}{3}$,
			$\frac{1}{6}$,$\frac{1}{6}$ respectively}
		\lCase{$4$}{$r_{i}' \leftarrow 0, 1,2,3,4,5$ with probability
			$\frac{1}{24}$,$\frac{1}{24}$,
			$\frac{1}{12}$,$\frac{1}{6}$,
			$\frac{1}{3}$,$\frac{1}{3}$ respectively}
		\lCase{$5$}{$r_{i}' \leftarrow 0, 1,2,3,4,5$ with probability
			$\frac{1}{48}$,$\frac{1}{48}$,
			$\frac{1}{24}$,$\frac{1}{12}$,
			$\frac{1}{6}$,$\frac{2}{3}$ respectively}
		\EndMatch
                \State{\textbf{if} $r'_i \geq t'$ \textbf{then} 
                  \textbf{halt} with $a_i = \top$ \textbf{else}
                  $a_i = \bot$}
\hide{
		\If{$r_{i}'>t'$}
		\State{$a_i=\top$}
		\State{\textbf{halt}}
		\Else
		\State{$a_i=\bot$}
		\EndIf
}
		\EndFor
		\EndProcedure
		
	\end{algorithmic}
	\caption{Input: private database $d$, 
          queries $f_i : d \rightarrow \{ 0, 1,
          \ldots, 5 \}$ with sensitivity $1$, threshold $t \in \{ 0,
          1, \ldots, 5 \}$; Output: $a_1, a_2, \ldots$}
	\label{algorithm:online-model}
\end{algorithm}

Assume that we have a threshold $t \in \{ 0, 1, \ldots, 5 \}$
and queries $\{ f_i : \Delta
(f_i) = 1 \}$. In order to protect privacy, our mechanism applies the
truncated $\frac{1}{4}$-geometric mechanism to obtain a perturbed
threshold $t'$. For each query $f_i$, the truncated $\frac{1}{2}$-geometric
mechanism is applied to its result $r_i = f_i(x)$. If the perturbed result
$r'_i$ is not less
than the perturbed threshold $t'$, the mechanism halts with the output
$\top$. Otherwise, it outputs $\bot$ and continues to the next query
(Algorithm~\ref{algorithm:online-model}). The above threshold
mechanism outputs a sequence of the form $\bot^* \top$. On
similar datasets, we want to show that the above
threshold mechanism outputs the same sequence with similar
probabilities. 

\begin{figure}
  \centering
    \resizebox{.8\columnwidth}{!}{
    \begin{tikzpicture}[->,>=stealth',shorten >=1pt,auto,node
      distance=2cm,node/.style={circle,draw}]
      \node[node] (t0r1) at (0, 4.25) { $t_0r_1$ };

      \draw (2.9, 3.45) rectangle (4.9, .05);
      \node at (4.65, 2.75) { $\cdots $};
      \node[node] (t'0r1) at (3.9, 2.75) { $t'_0r_1$ };
      \node at (3.2, 2.75) { $\cdots $};
      \draw (-2.8, 3.45) rectangle (2.8, .05);
      \node[node] (t'1r0) at (2.1, 2.75) { $t'_1r_0$ };
      \node[node] (t'1r1) at (0, 2.75) { $t'_1r_1$ };
      \node[node] (t'1r2) at (-2.1, 2.75) { $t'_1r_2$ };
      \node at (0, 1.85) { $\vdots$ };
      \node at (-2.1, 1.85) { $\vdots$ };
      \draw (-2.9, 3.45) rectangle (-4.9, .05);
      \node at (-3.15, 2.75) { $\cdots$ };
      \node[node] (t'2r1) at (-3.9, 2.75) { $t'_2r_1$ };
      \node at (-4.6, 2.75) { $\cdots$ };

      \node[node] (t'1r'0) at (2.1, .75) { $t'_1r'_0$ };
      \node[node] (t'1r'1) at (0, .75) { $t'_1r'_1$ };
      \node[node] (t'1r'2) at (-2.1, .75) { $t'_1r'_2$ };

      \node[node] (tt0rr0) at (2.1, -4.25) { $\underline{t}_0\underline{r}_0$ };

      \draw (2.9, -3.45) rectangle (4.9, -.05);
      \node[node] (tt'0rr0) at (4.2, -2.75) { $\underline{t}'_0\underline{r}_0$ };
      \node at (3.35, -2.75) { $\cdots$ };
      \draw (.8, -3.45) rectangle (2.8, -.05);
      \node[node] (tt'1rr0) at (2.1, -2.75) { $\underline{t}'_1\underline{r}_0$ };
      \node at (1.25, -2.75) { $\cdots$ };

      \draw (.7, -3.45) rectangle (-4.9, -.05);
      \node[node] (tt'2rr0) at (0, -2.75) { $\underline{t}'_2\underline{r}_0$ };
      \node[node] (tt'2rr1) at (-2.1, -2.75) { $\underline{t}'_2\underline{r}_1$ };
      \node[node] (tt'2rr2) at (-4.2, -2.75) { $\underline{t}'_2\underline{r}_2$ };

%      \node at (0, -1.65) { $\vdots$ };
      
      \node[node] (tt'2rr'0) at (0, -.75) { $\underline{t}'_2\underline{r}'_0$ };
      \node[node] (tt'2rr'1) at (-2.1, -.75) { $\underline{t}'_2\underline{r}'_1$ };
      \node at (-4.2, -1.65) { $\vdots$ };

      \node[node] (tt'2rr'2) at (-4.2, -.75) { $\underline{t}'_2\underline{r}'_2$ };


      \path
      (t0r1) edge node [above] {  } (t'0r1)
      (t0r1) edge node [left] {  } (t'1r1)
      (t0r1) edge node [above] {  } (t'2r1)

%      (t'1r1) edge node [above] { $\frac{1}{1}$ } (t'1r'0)
%      (t'1r1) edge node [above] { $\frac{1}{1}$ } (t'1r'1)
%      (t'1r1) edge node [above] { $\frac{1}{1}$ } (t'1r'2)

      (t'1r'0) edge [double] node [right]
              { \rotatebox{270}{$0\underline{0}, 0\underline{1}$} } (t'1r0)
      (t'1r'0) edge [double] node [right=-8] 
              { \rotatebox{315}{$1\underline{0}, 
                                 1\underline{1}, 1\underline{2}$} } (t'1r1)
      (t'1r'0) edge [double] node [right] 
              { \rotatebox{334}{$2\underline{1}, 2\underline{2}$} } (t'1r2)

      (tt0rr0) edge node [below] {  } (tt'0rr0)
      (tt0rr0) edge node [left] {  } (tt'1rr0)
      (tt0rr0) edge node [below] {  } (tt'2rr0)

      (tt'2rr'0) edge [double] node [right] 
                 { \rotatebox{270}{$0\underline{0}, 1\underline{0}$} } (tt'2rr0)
      (tt'2rr'0) edge [double] node [above] 
                 { \hide{$0|1, 1|1, 2|1$} } (tt'2rr1)
      (tt'2rr'0) edge [double] node [above] 
                 { \hide{$1|2, 2|2$} } (tt'2rr2)

      (tt'2rr'1) edge [double] node [right=15,below=-11]
                { \rotatebox{315}{$0\underline{0}, 1\underline{0}$} } (tt'2rr0)
      (tt'2rr'1) edge [double] node [above]
                { \hide{$0|1, 1|1, 2|1$} } (tt'2rr1)
      (tt'2rr'1) edge [double] node [above]
                { \hide{$1|2, 2|2$} } (tt'2rr2)

      ;
      \end{tikzpicture}
    }
  \caption{Markov Decision Process for Above Threshold}
  \label{figure:mdp-above-threshold}
\end{figure}

It is not hard to model the above threshold mechanism as a Markov
decision process (Figure~\ref{figure:mdp-above-threshold}). In the
figure, we sketch the model where the threshold and query results are
in $\{ 0, 1, 2 \}$. The model simulates two computation in parallel:
one for the dataset, the other for its neighbor. The state $t_ir_j$
represents the input threshold $i$ and the first query result
$j$; the state $t'_ir'_j$ represents the perturbed threshold $i$ and
the perturbed query result $j$. Other states are similar. Consider the
state $t_0r_1$. After applying the truncated $\frac{1}{4}$-geometric
mechanism, it goes to one of the states $t'_0r_1$, $t'_1r_1$, $t'_2r_1$
accordingly. From the state $t'_1r_1$, for instance, it moves to one
of $t'_1r'_0$, $t'_1r'_1$, $t'_1r'_2$ by applying the truncated
$\frac{1}{2}$-geometric mechanism to the query result. If it arrives
at $t'_1r'_1$ or $t'_1r'_2$, the perturbed query result is not less
than the perturbed threshold. The model halts with the output $\top$
by entering the state with a self loop. 
Otherwise, it moves to one of $t'_1r_0$, $t'_1r_1$, or $t'_1r_2$
non-deterministically (double arrows). The computation of
its neighbor is similar. We just use the underlined symbols to
represent threshold and query results. For instance, the state
$\underline{t}'_2\underline{r}'_1$ represents the perturbed threshold
$2$ and the perturbed query result $1$ in the neighbor.

Now, the non-deterministic choices in the two computation cannot be 
independent. Recall that the sensitivity of each query is
$1$. If the top computation moves to the state, say, $t'_1r_0$, it
means the next query result on the dataset is $0$. Subsequently, the
bottom computation can only move to $\underline{t}'_j\underline{r}_0$
or $\underline{t}'_j\underline{r}_1$ depending on its perturbed
threshold. This is where actions are useful. Define the actions
$\{ m\underline{n} : |m - n| \leq 1 \}$. The action
$m\underline{n}$ represents that the next query result for the dataset
and its neighbor are $m$ and $n$ respectively. For instance, the
non-deterministic choice from $t'_1r'_0$ to $t'_1r_0$ is associated
with two actions $0\underline{0}$ and $0\underline{1}$ (but not
$0\underline{2}$). Similarly, the choice from
$\underline{t}'_2\underline{r}'_1$ to
$\underline{t}'_2\underline{r}_0$ is associated with the actions
$0\underline{0}$ and $1\underline{0}$ (but not $2\underline{0}$). 
Assume the perturbed thresholds of the top and bottom computation are
$i$ and $j$ respectively. On the action $0\underline{0}$, the top
computation moves to $t'_ir_0$ and 
the bottom computation moves to $\underline{t}'_j\underline{r}_0$.
Actions make sure the two computation of neighbors
is modeled properly. Now consider the action sequence
$-,-,0\underline{1},-,2\underline{2},-,2\underline{1}$ from the states
$t_0r_1$ and $\underline{t}_0\underline{r}_0$ (``$-$'' represents the
purely probabilistic action). Together with the first query results, 
it denotes four consecutive query results $1$, $0$, $2$, $2$ on the
top computation, and $0$, $1$, $2$, $1$ on the bottom computation.
Each action sequence models two sequences of query results: one on the
top, the other on the bottom computation. Moreover, the difference of
the corresponding query results on the two computation is at most one
by the definition of the action set. Any sequence of adaptive query
results is hence formalized by an action sequence in our model.

It remains to define the neighborhood
relation. Recall the sensitivity is $1$. Consider the neighborhood
relation 
$\{ (t_ir_m, t_ir_m),
    (\underline{t}_i\underline{r}_n, \underline{t}_i\underline{r}_n),
    (t_ir_m, \underline{t}_i\underline{r}_n), 
    (\underline{t}_i\underline{r}_n, t_ir_m) : | m - n | \leq 1 \}$.
That is, two states are neighbors if they represent two inputs of the
same threshold and query results with difference at most one.