
Consider
\begin{eqnarray*}
  \Phi & ::= & \top \ |\ p \ |\ \neg \Phi \ |\ \Phi \wedge \Phi \ |\
               \PJ{J} \phi \ |\ \dpriv{\epsilon}{\delta} \phi\\
  \phi & ::= & \Phi \ |\ \neg\phi  \ |\  \phi\wedge\phi  \ |\  \gX \phi \ |\ \phi \until \phi
\end{eqnarray*}
\lz{Bow-Yaw: we use $\X$ or $\gX$?}
A \emph{state} formula $\Phi$ is either $\top$, an atomic proposition
$p$, the negation of a sate formula, the conjunction of two state
formulae, a \emph{probabilistic} operator $\PJ{J}$ with $J$
an interval in $[0, 1]$ followed by a path formula, or a
\emph{differentially private} operator $\dpriv{\epsilon}{\delta}$
 with two non-negative real numbers $\epsilon$ and $\delta$ followed
 by a path formula. A
\emph{path} formula $\phi$ is a \emph{next} operator ($\X$)
followed by a state formula, or an \emph{until} operator
($\until$) operator enclosed by two state formulae.

