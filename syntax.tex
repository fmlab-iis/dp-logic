
The syntax of $\dpCTL$ state and path formulae is given by:
\begin{eqnarray*}
  \Phi & ::= & p \ |\ \neg \Phi \ |\ \Phi \wedge \Phi \ |\
               \PJ{J} \phi \ |\ \dpriv{\epsilon}{\delta} \phi\\
  \phi & ::= & \Phi \ |\ \neg\phi  \ |\  \phi\wedge\phi  \ |\  \X \phi \ |\ \phi \until \phi
\end{eqnarray*}
A \emph{state} formula $\Phi$ is either  an atomic proposition
$p$, the negation of a sate formula, the conjunction of two state
formulae, a \emph{probabilistic} operator $\PJ{J}$ with $J$
an interval in $[0, 1]$ followed by a path formula, or a
\emph{differentially private} operator $\dpriv{\epsilon}{\delta}$
 with two non-negative real numbers $\epsilon$ and $\delta$ followed
 by a path formula. A
\emph{path} formula $\phi$ is simply a LTL formula, with temporal operator \emph{next}  ($\X$)
and  \emph{until} operator
($\until$) operator enclosed by two path formulae.
We often use syntactic sugar $F$ and $G$ defined by: $F \phi:= true \until \phi$ and $G\phi :=\neg F (\neg\phi)$.

As in the classical setting, we consider the sublogic $\dpCTL$ by allowing only path formulae of the form $\X\Phi$ and $\Phi\until\Phi$.
Moreover, one obtains $\PCTL$~\cite{pHanssonJ94} and $\PCTLstar$~\cite{BiancoA95} from $\dpCTL$ and $\dpCTLstar$
by removing the differentially private operator  $\dpriv{\epsilon}{\delta}$. 

