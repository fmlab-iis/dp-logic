The logic $\dpCTLstar$ can be interpreted over MDPs. Semantics over Markov chains generalizes to MDPs as well. %Note that two states with different enabled actions
%are trivially distinguishable; no privacy can be preserved.
Let $M = (S, \Act, \wp, L)$ be an MDP. Define the satisfaction
relation $M, \neighbor{S}, s \models \Phi$ for $\PJ{J}$ and $\dpriv{\epsilon}{\delta} \phi$ (others are standard) as follows.
\begin{eqnarray*}
%  M, \neighbor{S}, s \models \top\\
%  M, \neighbor{S}, s \models p  & \textmd{ if } &  p \in L(s)\\
%  M, \neighbor{S}, s \models \neg \Phi  & \textmd{ if } &  M, \neighbor{S}, s \not\models \Phi\\
%  M, \neighbor{S}, s \models \Phi_0 \wedge \Phi_1  & \textmd{ if } &  M, \neighbor{S}, s \models \Phi_0 \textmd{ and }
%  M, \neighbor{S}, s \models \Phi_1\\
  M, \neighbor{S}, s \models \PJ{J} \phi
  & \textmd{ if } &
  \myPr{s}{M_{\scheduler{S}}}{\neighbor{S}}{\phi} \in J
  \textmd{ for every scheduler } \scheduler{S}\\
  M, \neighbor{S}, s \models \dpriv{\epsilon}{\delta} \phi
  & \textmd{ if } &
  \textmd{for all } t \textmd{ with } s \neighbor{S} t \textmd{ and
   query scheduler } \scheduler{Q},
   \myPr{s}{M_{\scheduler{Q}}}{\neighbor{S}}{\phi} \leq
  e^{\epsilon} \cdot
   \\
  & &
   \myPr{t}{M_{\scheduler{Q}}}{\neighbor{S}}{\phi} + \delta
   \textmd{ and }
   \myPr{t}{M_{\scheduler{Q}}}{\neighbor{S}}{\phi} \leq e^{\epsilon} \cdot
   \myPr{s}{M_{\scheduler{Q}}}{\neighbor{S}}{\phi} + \delta
\end{eqnarray*}

Recall that $M_{\scheduler{S}}$ is but a Markov chain. The semantics
of $M_{\scheduler{S}}, \neighbor{S}, \pi \models \phi$ and hence the
probability $\myPr{s}{M_{\scheduler{S}}}{\neighbor{S}}{\phi}$ are
defined as in Markov chains.
The semantics of $\dpCTLstar$ on MDPs
is again standard except the differentially private operator
$\dpriv{\epsilon}{\delta}$. For any path formula $\phi$,
$\dpriv{\epsilon}{\delta} \phi$ specifies states whose probability
of having paths satisfying $\phi$ are $(\epsilon, \delta)$-close to
those of all its neighbors for query schedulers. That is, no
query scheduler can force any of neighbors to have
probabilistically distinguishable behaviors.

\noindent
\paragraph{Justification of query schedulers.}
We use query schedulers in the semantics for the differentially
private operator. A definition with history-dependent schedulers
might be
\begin{eqnarray*}
  M, \neighbor{S}, s \models \dpriv{\epsilon}{\delta}^{\mathit{bad}} \phi
  & \textmd{ if } &
  \textmd{for all } t \textmd{ with } s \neighbor{S} t \textmd{ and
  scheduler } \scheduler{S},
  \myPr{s}{M_{\scheduler{S}}}{\neighbor{S}}{\phi} \leq
  e^{\epsilon} \cdot\\
  && \myPr{t}{M_{\scheduler{S}}}{\neighbor{S}}{\phi} + \delta \textmd{ and }
  \myPr{t}{M_{\scheduler{S}}}{\neighbor{S}}{\phi} \leq
  e^{\epsilon} \cdot \myPr{s}{M_{\scheduler{S}}}{\neighbor{S}}{\phi}
  + \delta
\end{eqnarray*}
A state satisfies $\dpriv{\epsilon}{\delta}^{\mathit{bad}} \phi$ if
no history-dependent scheduler could differentiate the probabilities
of haveing paths satisfying
$\phi$ from neighbors. Recall that a history-dependent scheduler takes
actions by previous states. Such a definition would allow
schedulers to take different actions from different states. 
Two neighbors could be differentiated by different action sequences
erroneously.
A query scheduler $\scheduler{Q} : \bbfZ^{\geq 0} \rightarrow \Act$,
on the other hand, corresponds to a query sequence. A state satisfies
$\dpriv{\epsilon}{\delta} \phi$ if no query sequence can differentiate
the probabilities of having paths satisfying $\phi$ from neighbors.
Our semantics agrees with the informal interpretation of differential
privacy for such systems. We hence
prefer the original definition.

\subsection{Model Checking}
For an MDP $M =
(S, \Act, \wp, L)$, we want to compute $\{ s : M, \neighbor{S}, s \models
\dpriv{\epsilon}{\delta} \phi \}$ . Recall the semantics of $\dpriv{\epsilon}{\delta}
\phi$. Given $s, t$ with $s \neighbor{S} t$ and a path formula
$\phi$, we need to decide whether
$\myPr{s}{M_{\scheduler{Q}}}{\neighbor{S}}{\phi} \leq
e^{\epsilon} \myPr{t}{M_{\scheduler{Q}}}{\neighbor{S}}{\phi} + \delta$
for every query scheduler $\scheduler{Q}$.
If the path formula only contains next operators, it is easy to deal with. When $\phi$ is $\gX B$ with $B\subseteq S$, only one step behavior needs to be considered. Thus only the first action in the query sequence is needed. It can also be easily generalized to nested next operators:  one needs only to enumerate
all actions query sequences of length obtained by the number of nested $\X$ operators.
The problem however is undecidable in general.

\begin{theorem}\label{theorem:mdp-model-checking}
The $\dpCTLstar$ model checking problem for MDPs  is undecidable.
\end{theorem}

The proof is in Appendix. We discuss some decidable special cases. Consider the formula $\phi:=F B$ with $B\subseteq S$ and assume that states in $B$ are absorbing (with only self-loops). For the case $\epsilon=0$, the condition reduces to
$\myPr{s}{M_{\scheduler{Q}}}{\neighbor{S}}{F B} -
\myPr{t}{M_{\scheduler{Q}}}{\neighbor{S}}{F B}\le \delta$. If $\delta=0$ it is the classical language equivalence problem for probabilistic automata~\cite{Rabin63}, which can be solved in polynomial time. However, if $\delta>0$, the problem
becomes an approximate version of the language equivalence problem. To the best of our knowledge, the decision problem is still open, and has only be investigated for the special case when all states are connected~\cite{Tzeng92}.

\hide{
Since the scheduler attaining the maximally probable
behavior from a state may be different from the scheduler attaining
the maximally probable behavior from its neighbors, two
neighbors may still be distinguished by a scheduler. The
weaker definition does not preserve differential privacy. We hence
prefer the original definition.
}

Despite of the negative result in Theorem~\ref{theorem:mdp-model-checking},
a sufficient condition for $M, \neighbor{S}, s \models
\dpriv{\epsilon}{\delta} \phi$ is available.
To see this, observe that for any state $s \in S$ and
query scheduler $\scheduler{Q}$, we have
\[
  \min_{\scheduler{S}} \myPr{s}{M_{\scheduler{S}}}{\neighbor{S}}{\phi} 
  \leq
  \myPr{s}{M_{\scheduler{Q}}}{\neighbor{S}}{\phi} 
  \leq
  \max_{\scheduler{S}} \myPr{s}{M_{\scheduler{S}}}{\neighbor{S}}{\phi} 
\]
where the minimum and maximum are taken over all schedulers
$\scheduler{S}$. Hence, 
\begin{eqnarray*}
  \myPr{s}{M_{\scheduler{Q}}}{\neighbor{S}}{\phi} -
      e^\epsilon \myPr{t}{M_{\scheduler{Q}}}{\neighbor{S}}{\phi}
  \leq
    \max_{\scheduler{S}} \myPr{s}{M_{\scheduler{S}}}{\neighbor{S}}{\phi}
    - e^\epsilon \cdot
    \min_{\scheduler{S}} \myPr{t}{M_{\scheduler{S}}}{\neighbor{S}}{\phi}
\end{eqnarray*}
for any $s, t \in S$ and query scheduler $\scheduler{Q}$. We have the
following proposition:
\begin{proposition}\label{proposition:sufficient-condition-mdp-model-checking}
  Let $M = (S, \Act, \wp, L)$ be an MDP, $\neighbor{S}$ a
  neighborhood relation on $S$.
  $M, \neighbor{S}, s \models \dpriv{\epsilon}{\delta} \phi$ if
  for any $s, t \in S$ with $s \neighbor{S} t$, we have
  $\max\limits_{\scheduler{S}} \myPr{s}{M_{\scheduler{S}}}{\neighbor{S}}{\phi}
    - e^\epsilon \cdot
    \min\limits_{\scheduler{S}} \myPr{t}{M_{\scheduler{S}}}{\neighbor{S}}{\phi}
    \leq \delta$, and
  $\max\limits_{\scheduler{S}} \myPr{t}{M_{\scheduler{S}}}{\neighbor{S}}{\phi}
    - e^\epsilon \cdot
    \min\limits_{\scheduler{S}} \myPr{s}{M_{\scheduler{S}}}{\neighbor{S}}{\phi}
    \leq \delta$.
\end{proposition}

For any $s \in S$, recall that $\max\limits_{\scheduler{S}}
\myPr{s}{M_{\scheduler{S}}}{\neighbor{S}}{\phi}$ and
$\min\limits_{\scheduler{S}}
\myPr{s}{M_{\scheduler{S}}}{\neighbor{S}}{\phi}$ can be efficiently 
computed~\cite{BK:08:PMC}. 
Using
Proposition~\ref{proposition:sufficient-condition-mdp-model-checking},
$M, \neighbor{S}, s \models \dpriv{\epsilon}{\delta} \phi$ can be
checked soundly and efficiently.

We model Algorithm~\ref{algorithm:online-model} as in Section~\ref{subsec:threshold} 
and apply implementation on model checking to check whether the
mechanism is differentially private. An advantage of model checking is that we
can check property of any exact value of parameters $\epsilon$ and $\delta$.
On the one hand,
smaller values for particular neighborhood database could be obtained.
For instance, for the state $t_3r_5$ and its neighborhood state $\underline{t}_3\underline{r}_4$, we verify the state
satisfies the property $\bigwedge_{k \in \bbfZ^{\geq 0}} \dpriv{0}{0.17} ((\X^k\bot) \top)$(Note that with k rising, the reachability probability decreases to approximately 0).
On the other hand, we could verify that for all the neighborhood database, i.e, the neighborhood states, the property $\bigwedge_{k \in \bbfZ^{\geq 0}} \dpriv{1}{0.74} ((\X^k\bot) \top)$ is satisfied. This indicates that the mechanism in Algorithm~\ref{algorithm:online-model} is differential private with $\epsilon = 1$ and $\delta = 0.74$. The parameter $\delta$ here seems large, due to the fact that 
Propsition~\ref{proposition:sufficient-condition-mdp-model-checking} is a sufficient 
condition.





