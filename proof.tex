\begin{proof}
The proof follows by a reduction from the \emph{emptiness} problem for probabilistic automata.
A \emph{probabilistic automaton}~\cite{Rabin63} is a tuple $\mathcal{A}= (S,\Sigma,M,s_0,B)$ where
\begin{itemize}
  \item $S$ is a finite set of states,
  \item $\Sigma$ is the finite set of input alphabet,
  \item $M: S\times\Sigma\times S \to [0,1]$ such that $\sum_{t\in S} M(s,\alpha,t)=1$ for all $s\in S$ and $\alpha\in \Sigma$,
  \item $s_0\in S$ is the initial state,
  \item $B\subseteq S$ is a set of accepting states.
\end{itemize}

Each input alphabet $\alpha$ induces a stochastic matrix $M(\alpha)$ in the obvious way. Let $\lambda$ denote the empty string. For $\eta\in\Sigma^*$ we define $M(\eta)$ inductively by: $M(\lambda)$ is the identity matrix, $M(x\eta')=M(x)M(\eta')$. Thus, $M(\eta)(s,s')$ denotes the probability of going from $s$ to $s'$ after reading $\eta$. Let $v_B$ denote the characteristic row vector for the set $B$, and $v_{s_0}$ denote the characteristic row vector for the set $\{s_0\}$. Then, the accepting probably of $\eta$ by $\mathcal{A}$ is defined as $v_{s_0}\cdot M(\eta)\cdot (v_B)^c$ where $(v_B)^c$ denotes the transpose of $v_B$. The following \emph{emptiness problem} is know to be undecidable~\cite{Paz71}:
%~\cite{Worrell08}:

\paragraph{Emptiness problem:} Given a probabilistic automaton $\mathcal{A}= (S,\Sigma,M,s_0,B)$, whether there exists $\eta\in\Sigma^*$ such that $v_{s_0}\cdot M(\eta)\cdot (v_B)^c > 0$?

Now we establish the proof by reducing the threshold problem to our $\dpCTLstar$ model checking problem. Given the probabilistic automaton  $\mathcal{A}= (S,\Sigma,M,s_0,B)$, assume we have a \emph{primed} copy $\mathcal{A}'= (S',\Sigma,M',s_0',\emptyset)$.

Let $AP:=\{at_B\}$. Now we construct our MDP  $M =
(S\cupdot S', \Sigma, \wp, L)$ where $\wp(s,a,t)$ equals to $M(s,a,t)$ if $s,t\in S$ and to $M'(s,a,t)$ if $s,t\in S'$. We define the neighbor relation $N_S:=\{(s_0,s_0'),(s_0',s_0)\}$ by relating states $s_0,s_0'$. The labelling function $L$ is defined by $L(s)=\{at_B\}$ if $s\in B$ and $L(s)=\emptyset$ otherwise.

Now we consider the formula  $\Phi = \dpriv{1}{0} (F at_B)$. For the reduction we prove $s_0\models \dpriv{1}{0} (F at_B)$ iff for all $\eta\in\Sigma^*$ it holds $v_{s_0}\cdot M(\eta)\cdot (v_B)^c \leq 0$.

First we assume $s_0\models \dpriv{1}{0} (F at_B)$. By $\dpCTLstar$ semantics we have that for all query scheduler $\scheduler{Q}\in\Sigma^\omega$,
$\myPr{s_0}{M_{\scheduler{Q}}}{\neighbor{S}}{F at_B} \leq
  e \cdot
  \myPr{s_0'}{M_{\scheduler{Q}}}{\neighbor{S}}{F at_B}$. Since the set of accepting state in the primed copy is empty, we have $\myPr{s_0'}{M_{\scheduler{Q}}}{\neighbor{S}}{F at_B}=0$, thus we have
$\myPr{s_0}{M_{\scheduler{Q}}}{\neighbor{S}}{F at_B} \leq
 0$. This implies  $v_{s_0}\cdot M(\eta)\cdot (v_B)^c \leq 0$ for all $\eta\in\Sigma^*$.

 For the other direction, assume that all $\eta\in\Sigma^*$ it holds $v_{s_0}\cdot M(\eta)\cdot (v_B)^c \leq 0$. We prove by contradiction. Assume that $s_0\not\models \dpriv{1}{0} (F at_B)$. Since the relation $N_S$ contains only one pair, it implies that there exists a query scheduler $\scheduler{Q}\in\Sigma^\omega$ such that $\myPr{s_0}{M_{\scheduler{Q}}}{\neighbor{S}}{F at_B} >0$. It is then easy to construct a finite sequence $\eta\in\Sigma^*$ with  $v_{s_0}\cdot M(\eta)\cdot (v_B)^c > 0$, a contradiction.
\end{proof}
