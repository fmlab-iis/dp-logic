
Given a finite discrete-time Markov chain $K = (S, \wp, L)$, a
\emph{neighborhood relation on $S$} $\neighbor{S} \subseteq S \times S$
is a reflexive and symmetric relation on $S$. We will write $s
\neighbor{S} t$ when $(s, t) \in \neighbor{S}$. If $s \neighbor{S} t$,
we say $s$ and $t$ are \emph{neighbors} or $t$ is a \emph{neighbor} of
$s$. Define the satisfaction relation $K, \neighbor{S}, s
\models \Phi$ as follows.
\begin{eqnarray*}
  K, \neighbor{S}, s \models \top\\
  K, \neighbor{S}, s \models p
  & \textmd{ if } &
  p \in L(s)\\
  K, \neighbor{S}, s \models \neg \Phi
  & \textmd{ if } &
  K, \neighbor{S}, s \not\models \Phi\\
  K, \neighbor{S}, s \models \Phi_0 \wedge \Phi_1
  & \textmd{ if } &
  K, \neighbor{S}, s \models \Phi_0 \textmd{ and }
  K, \neighbor{S}, s \models \Phi_1\\
  K, \neighbor{S}, s \models \PJ{J} \phi
  & \textmd{ if } &
  \Pr [\{ \pi : K, \neighbor{S}, \pi \models \phi \textmd{ with }
                    \pi_0 = s\}] \in J \\
  K, \neighbor{S}, s \models \dpriv{\epsilon}{\delta} \phi
  & \textmd{ if } &
  \textmd{for every } t \textmd{ with }  s \neighbor{S} t,
      p(s) \leq e^{\epsilon} p(t) + \delta \textmd{ and }
      p(t) \leq e^{\epsilon} p(s)  +  \\
  & &  \delta \textmd{ where } p(x) = \Pr [\{
      \pi : K, \neighbor{S}, \pi \models \phi \textmd{ with}
      \pi_0 = x \}]
\end{eqnarray*}

Moreover, the relation $K, \neighbor{S}, \pi \models \phi$ is defined as
for the standard LTL formulas. We only give the semantics for $\X$ and $\until$ as follows.
\begin{eqnarray*}
  K, \neighbor{S}, \pi \models \X \phi
  & \textmd{ if } &
  K, \neighbor{S}, \pi[1] \models \phi\\
  K, \neighbor{S}, \pi \models \phi \until \psi
  & \textmd{ if } &
  \textmd{there is a } j \geq 0 \textmd{ such that }
  K, \neighbor{S}, \pi[j] \models \psi \textmd{ and } \\
  & & K, \neighbor{S}, \pi[k] \models \phi
      \textmd{ for every } 0 \leq k < j
\hide{
  K, \neighbor{S}, \pi \models \Phi \buntil{n} \Psi
  & \textmd{ if } &
  \textmd{there is a } 0 \leq j \leq n \textmd{ such that }
  K, \neighbor{S}, \pi[j] \models \Psi \textmd{ and } \\
  & & K, \neighbor{S}, \pi[k] \models \Phi
      \textmd{ for every } 0 \leq k < j
}
\end{eqnarray*}

To simplify our notation, for any Markov chain $K$, neighborhood
relation $N$ on $S$, a state $s \in S$, and a path formula $\phi$,
define
\[
\myPr{s}{K}{N}{\phi} =
\Pr [ \{ \pi : K, N, \pi \models \phi \textmd{ with } \pi_0 = s \} ].
\]
That is, $\myPr{s}{K}{N}{\phi}$ denotes the probability of paths
satisfying $\phi$ from $s$ on $K$ with $N$. The semantics of the
differentially private operator hence becomes
\begin{eqnarray*}
  K, \neighbor{S}, s \models \dpriv{\epsilon}{\delta} \phi
  & \textmd{ if } &
  \textmd{for every } t \textmd{ with }  s \neighbor{S} t,
      \myPr{s}{K}{N}{\phi} \leq e^{\epsilon} \myPr{t}{K}{N}{\phi} + \delta
      \textmd{ and }\\
  & &  \myPr{t}{K}{N}{\phi} \leq e^{\epsilon} \myPr{s}{K}{N}{\phi} + \delta
\end{eqnarray*}

Other than the differentially private operator, the semantics of
$\dpCTLstar$ is standard~\cite{BK:08:PMC}.
To intuit the semantics of $\dpriv{\epsilon}{\delta} \phi$,
recall that  $\myPr{s}{K}{N}{\phi}$ is the probability of having
paths satisfying $\phi$ from $s$. Thus, a state $s$ satisfies
$\dpriv{\epsilon}{\delta} \phi$ if having paths satisfying $\phi$
from $s$ is $(\epsilon, \delta)$-close to having paths satisfying
$\phi$ from every neighbor of $s$. In other words, it is
probabilistically similar to observe paths satisfying $\phi$ from $s$
and from all neighbors of $s$.

\subsection{Model Checking Algorithms}
We first describe the model checking algorithm for $\dpCTL$, which follows the classical
algorithms for $\PCTL$ by computing the states satisfying
sub state-formulae inductively~\cite{BK:08:PMC}. It hence suffices to
consider the inductive step where the states satisfying the subformula
$\dpriv{\epsilon}{\delta} (\phi)$ is computed.

%We start with the algorithm for Markov chains.

In the classical $\PCTL$ model checking algorithm for Markov chains,
states satisfying the subformula $\PJ{J} \phi$ are obtained by
computing $\myPr{s}{K}{\neighbor{S}}{\phi}$ for state $s \in S$.
These probabilities can be computed by solving linear equations or
through iterative approximations. We summarize it in the following
theorem~\cite{BK:08:PMC}:

\begin{lemma}
  Let $K = (S, \wp, L)$ be a Markov chain, $s
  \in S$, and $B, C \subseteq S$. The probabilities
  $\myPr{s}{K}{\neighbor{S}}{\gX B}$ and
  $\myPr{s}{K}{\neighbor{S}}{B \mho C}$
\hide{
  $\Pr[\{ \pi : K, \neighbor{K}, \pi \models B \buntil{n} C \textmd{
    with } \pi_0 = s \}]$
} are computable within time polynomial in
  $|S|$.
  \label{lemma:PJ-subroutines}
\end{lemma}

In Lemma~\ref{lemma:PJ-subroutines}, we abuse the notation slightly to
admit path formulae of the form $\gX B$ (next $B$) and
$B \guntil C$ ($B$ until $C$) with $B, C \subseteq S$ as
in~\cite{BK:08:PMC}. They are interpreted by introducing new atomic
propositions $B$ and $C$ for each $s \in B$ and $s \in C$
respectively. We will also use graphical symbols $\gF C$ ($C$ in
the future) and $\gG C$ (globally $C$) in such path formulae
$\varphi$ to minimize confusion.

In order to determine the set $\{ s : K, \neighbor{S}, s \models
\dpriv{\epsilon}{\delta} \varphi \}$, our algorithm first computes
the probabilities $p (s) = \myPr{s}{K}{\neighbor{S}}{\varphi}$ for every
$s \in S$ (Algorithm~\ref{algorithm:sat-dpriv-dtmc}). For each $s \in S$,
it then compares the probabilities $p (s)$ and $p (t)$ for every
neighbor $t$ of $s$. If there is a neighbor $t$ such that $p (s)$ and
$p (t)$ are not $(\epsilon, \delta)$-close, the state $s$
is removed from the result. Hence
Algorithm~\ref{algorithm:sat-dpriv-dtmc} returns all states which are
$(\epsilon, \delta)$-close to their neighbors.
The algorithm requires at most $O
(|S|^2)$ additional steps. We hence have the following results:

\begin{algorithm}
  \begin{algorithmic}[1]
    \Function{SAT}{$K$, $\neighbor{S}$, $\phi$}
    \Match{$\phi$}
    \Comment{by Lemma~\ref{lemma:PJ-subroutines}}
    \Case{$\X \Psi$}
      \State{$B \leftarrow \textmd{SAT} (K, \neighbor{S}, \Psi)$}
      \State{$p(s) \leftarrow \myPr{s}{K}{\neighbor{S}}{\gX B}$
        for every $s \in S$}
    \EndCase
    \Case{$\Psi \until \Psi'$}
      \State{$B \leftarrow \textmd{SAT} (K, \neighbor{S}, \Psi)$}
      \State{$C \leftarrow \textmd{SAT} (K, \neighbor{S}, \Psi')$}
      \State{$p(s) \leftarrow \myPr{s}{K}{\neighbor{S}}{B \guntil C}$ for
        every $s \in S$}
    \EndCase
    \EndMatch
    \State{$R \leftarrow S$}
    \For{$s \in S$}
      \For{$t$ with $s \neighbor{K} t$}
        \If{$p(s) \not\leq e^{\epsilon} p(t) + \delta$ or
            $p(t) \not\leq e^{\epsilon} p(s) + \delta$}
        {remove $s$ from $R$}
        \EndIf
      \EndFor
    \EndFor

    \State{\Return $R$}
    \EndFunction
  \end{algorithmic}
  \caption{SAT($\phi$, $\neighbor{K}$)}
  \label{algorithm:sat-dpriv-dtmc}
\end{algorithm}

\begin{proposition}
  Let $K = (S, \wp, L)$ be a Markov chain.
  $\{ s : K, \neighbor{K}, s \models \dpriv{\epsilon}{\delta} \phi \}$ is
  computable within time polynomial in $|S|$.
\end{proposition}

\begin{corollary}
  Let $K = (S, \wp, L)$ be a Markov chain and $\Phi$ a $\dpCTL$
  formula. $\{ s : K, \neighbor{K}, s \models \Phi \}$ is
  computable within time polynomial in $|S|$ and $|\Phi|$.
\end{corollary}

The model checking algorithm for $\dpCTLstar$ can be treated as in the classical setting \cite{BK:08:PMC}, as all we need is to  compute the  probability
  $\myPr{s}{K}{\neighbor{S}}{\phi}$ with general path formula $\phi$. For this purpose one first constructs a deterministic $\omega$-automaton $R$, such as deterministic Rabin automaton. Then, the probability reduces to a
  reachability probability in the product Markov chain obtained from $K$ and $R$. There are more efficient algorithms without the product construction, see \cite{} for details.
