
The sensitivity of queries is required to apply the (truncated)
$\alpha$-geometric mechanism. Recall that the sensitivity of a query
is the maximal difference of query results on any two neighbors. There are
two problems for such mechanisms depending on query sensitivity. First, 
sensitivity of queries can be hard to compute or estimate. Second,
the sensitivity over arbitrary neighbors can be too conservative for
the actual dataset in use. One therefore would like to have
mechanisms independent of query sensitivity.

Subsampling is a technique to design such mechanisms. Concretely, let
us consider $\mathcal{X} = \{ R, B \}$ (for red and blue teams) and a
dataset $d \in \mathcal{X}^n$. Suppose we would like to ask which team
is the majority in the dataset while respecting individual
privacy. This can be achieved as
follows~(Algorithm~\ref{algorithm:subsampling-majority}). The
mechanism first samples $m$ 
sub-datasets $\hat{d}_1, \hat{d}_2, \ldots, \hat{d}_m$ from
$d$~(line~\ref{algorithm:subsampling-majority-subsample}). 
It then computes the majority of each sub-dataset and obtains $m$
sub-results. Let $\mathit{count}_R$ and $\mathit{count}_B$ be the
number of sub-datasets with the majority $R$ and $B$
respectively~(line~\ref{algorithm:subsampling-majority-count}). 
Since there are $m$ sub-datasets, we have
$\mathit{count}_R + \mathit{count}_B = m$. To ensure differential
privacy, the mechanism makes sure the difference $| \mathit{count}_R -
\mathit{count}_B |$ is significantly large after perturbation. In
line~\ref{algorithm:subsampling-majority-noise},
$\mathit{Lap}(p)$ denotes a continuous random
variable with the Laplace distribution whose probability density
function is $f(x) = \frac{1}{2p}e^{-|x|/p}$.
If so,
it reports the majority of the $m$
sub-results~(line~\ref{algorithm:subsampling-majority-output}). Otherwise,
no information is
revealed~(line~\ref{algorithm:subsampling-majority-mute}).

\begin{algorithm}
  \begin{algorithmic}[1]
    \Function{SubsamplingMajority}{$d$, $f$}
    \Require{$d \in \{ R, B \}^n$, $f : \{ R, B \}^* \rightarrow \{ R,
      B \}$}
    \State{$q, m \leftarrow 
            \frac{\epsilon}{64\ln (1/\delta)},
            \frac{\log (n/\delta)}{q^2}$}
    \State{Subsample $m$ data sets $\hat{d}_1, \hat{d}_2, \ldots,
      \hat{d}_m$ from $d$ where each row of $d$ is chosen with
      probability $q$}
    \label{algorithm:subsampling-majority-subsample}
    \State{$\mathit{count}_R, \mathit{count}_B \leftarrow 
            | \{ i : f (\hat{d}_i) = R \} |,
            | \{ i : f (\hat{d}_i) = B \} |$}
    \label{algorithm:subsampling-majority-count}
    \State{$r \leftarrow | \mathit{count}_R - \mathit{count}_B | /
      (4mq) - 1$}
    \label{algorithm:subsampling-majority-difference}
    \If{${r} + \mathit{Lap}(\frac{1}{\epsilon}) > \ln (1/\delta)/\epsilon$}
    \label{algorithm:subsampling-majority-noise}
      \State{\textbf{if} $\mathit{count}_R \geq \mathit{count}_B$
             \textbf{then} \Return { $R$ } \textbf{else} \Return {$B$} } 
      \label{algorithm:subsampling-majority-output}
    \Else
      \State{\Return {$\bot$}}
      \label{algorithm:subsampling-majority-mute}
    \EndIf
  \EndFunction
  \end{algorithmic}
  \caption{Subsampling Majority}
  \label{algorithm:subsampling-majority}
\end{algorithm}

Fix the dataset size $n$ and privacy parameters $\epsilon$, $\delta$,
the subsampling majority mechanism can be modeled by a Markov chain.
Figure~\ref{figure:subsampling-majority-markov-chain} gives a sketch
of the Markov chain for $n = 3$. The leftmost four states represent
all possible datasets. From each dataset, we compute the probability
of having $(\mathit{count}_R, \mathit{count}_B) = (i, j)$ with $i + j
= m$ from $m$ sub-datasets. Since $| \mathit{count}_R -
\mathit{count}_B |$ can have only finitely many values, the number of
possible values of $r$ 
(line~\ref{algorithm:subsampling-majority-difference}) is in a finite
set $\{ r_m, \ldots, r_M \}$. For each $r \in \{ r_m, \ldots, r_M \}$,
the probability of having $r + \mathit{Lap}(\frac{1}{\epsilon}) > \ln
(1/\delta)/\epsilon$ (line~\ref{algorithm:subsampling-majority-noise})
is equal to the probability of $\mathit{Lap}(\frac{1}{\epsilon}) > \ln
(1/\delta)/\epsilon - r$. This is equal to $\int^{\infty}_{\ln
  (1/\delta)/\epsilon - r} \frac{\epsilon}{2}e^{-\epsilon|x|}$. 


\begin{figure}
  \centering
    \resizebox{.9\columnwidth}{!}{
    \begin{tikzpicture}[->,>=stealth',shorten >=1pt,auto,node
      distance=2cm,node/.style={circle,draw}]

      \node[node] (rrr) at (-5,  2.25) { $rrr$ };
%      \node at (-4.4, 2) { $\vdots$ };
      \node[node] (rrb) at (-5,  0.75) { $rrb$ };
%      \node at (-4.4, .75) { $\vdots$ };
      \node[node] (rbb) at (-5, -0.75) { $rbb$ };
%      \node at (-4.4, -.5) { $\vdots$ };
      \node[node] (bbb) at (-5, -2.25) { $bbb$ };
%      \node at (-4.4, -1.75) { $\vdots$ };

      \node[node] (rmb0) at (-2.5,  2.25) { $m,0$ };
      \node at (-2.5, 0) { $\vdots$ };
      \node[node] (r0bm) at (-2.5, -2.25) { $0,m$ };

      \node[node] (Rm) at (0,  1.5) { $r_m$ };
      \node at (0, 0) { $\vdots$ };
      \node[node] (RM) at (0, -1.5) { $r_M$ };

      \node[node] (top) at (2.5,  0.75) { $\top$ };
      \node[node] (bot) at (2.5, -0.75) { $\bot$ };

      \node[node] (outR) at (5,  1.25) { $R$ };
      \node[node] (outB) at (5, .25) { $B$ };

      \path
      (rrr) edge node [above] { } (rmb0)
      (rrr) edge node [above] { } (r0bm)
      (rrb) edge node [above] { } (rmb0)
      (rrb) edge node [above] { } (r0bm)
      (rbb) edge node [above] { } (rmb0)
      (rbb) edge node [above] { } (r0bm)
      (bbb) edge node [above] { } (rmb0)
      (bbb) edge node [above] { } (r0bm)

      (rmb0) edge node [above] {} (RM)
      (r0bm) edge node [above] {} (RM)

      (Rm) edge node [above] {} (top)
      (Rm) edge node [above] {} (bot)
      (RM) edge node [above] {} (top)
      (RM) edge node [above] {} (bot)

      (top) edge (outR)
      (top) edge (outB)
      ;

      \end{tikzpicture}
    }
  \caption{Markov Chain for Subsampling Majority}
  \label{figure:subsampling-majority-markov-chain}
\end{figure}
