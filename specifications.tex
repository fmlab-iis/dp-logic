In this section we describe how properties in the differential privacy literature can be expressed using $\dpCTLstar$ formulae.

\paragraph{Standard Differential Privacy.}
The formula $\dpriv{3}{0} (\X Y)$ holds in state $s$ if for  $t$ with $sNt$, the probability of satisfying $\X Y$ from $s$ and $t$ satisfy:
$\myPr{s}{K}{N}{\X Y}\le 3\myPr{t}{K}{N}{\X Y}$. The formula $\dpriv{3}{0} (\X Y)\wedge \dpriv{3}{0} (\X N)$ thus specifies the differential privacy property for states $+$ and $-$ in Example~\ref{exa:survey}.

 Define the $\dpCTL$ formula
$\psi = \dpriv{.5}{0} (\X out_0) \wedge \dpriv{.5}{0} (\X out_1) \wedge
\cdots \wedge \dpriv{.5}{0} (\X out_5)$. If the state $in_k$ satisfies
$\psi$ for $k = 0, \ldots, 5$, then the $.5$-geometric mechanism is
$(\ln 2, 0)$-differentially private.


\paragraph{Compositional Properties.}
Compositional aspect is one of the building block for differentially private algorithms. According to the compositional theorem \cite[Theorem 3.16]{DR:14:AFDP}, for $(\epsilon_1,\delta_1)$-differentially private algorithm $M_1$ and
$(\epsilon_2,\delta_2)$-differentially private algorithm $M_2$, their combination $M$ defined by $M(x)=(M_1(x), M_2(x))$ is $(\epsilon_1 + \epsilon_2,\delta_1+\delta_2)$-differentially private. The degradation is rooted in the repeated computation. To illustrate this property, we consider the extended survey example below where the curator allows two consequent queries.
The corresponding model is depicted in Figure~\ref{figure:2-dp-mc}, where states $Y_1,Y_2$ are labelled by $Y$, and $N_1,N_2$ are labelled by $N$.

\begin{figure}
    \centering
    \begin{subfigure}{.4\columnwidth}
\resizebox{\columnwidth}{!}{
    \begin{tikzpicture}[->,>=stealth',shorten >=1pt,auto,node
    distance=1cm,node/.style={circle,draw}]
    \node[node] (p) at ( -1.5,  0) { $+$ };
    \node[node] (Y1) at ( 0, .8) { $Y_1$ };
    \node[node] (N1) at ( 0, -.8) { $N_1$ };
    \node[node] (Y2) at (1.5,  .8) { $Y_2$ };
    \node[node] (N2) at ( 1.5, -.8) { $N_2$ };

    \path
    (p) edge node [above] { $\frac{3}{4}$ } (Y1)
    (p) edge node [below] { $\frac{1}{4}$ } (N1)
    (Y1) edge node [above] { $\frac{3}{4}$ } (Y2)
    (Y1) edge node [above] { $\frac{1}{4}$ } (N2)
    (N1) edge node [below] { $\frac{3}{4}$ } (Y2)
    (N1) edge node [below] { $\frac{1}{4}$ } (N2)
    ;
    \end{tikzpicture}
}
%        \caption{Markov Chain of }
%        \label{figure:2-dp-mc1}
    \end{subfigure}
    \hspace{.05\columnwidth}
    \begin{subfigure}{.4\columnwidth}
        \resizebox{\columnwidth}{!}{
            \begin{tikzpicture}[->,>=stealth',shorten >=1pt,auto,node
    distance=1cm,node/.style={circle,draw}]
    \node[node] (n) at ( -1.5,  0) { $-$ };
    \node[node] (Y1) at ( 0, .8) { $Y_1$ };
    \node[node] (N1) at ( 0, -.8) { $N_1$ };
    \node[node] (Y2) at (1.5,  .8) { $Y_2$ };
    \node[node] (N2) at ( 1.5, -.8) { $N_2$ };

    \path
    (n) edge node [above] { $\frac{1}{4}$ } (Y1)
    (n) edge node [below] { $\frac{3}{4}$ } (N1)
    (Y1) edge node [above] { $\frac{1}{4}$ } (Y2)
    (Y1) edge node [above] { $\frac{3}{4}$ } (N2)
    (N1) edge node [below] { $\frac{1}{4}$ } (Y2)
    (N1) edge node [below] { $\frac{3}{4}$ } (N2)
    ;
            \end{tikzpicture}
        }
%        \caption{Markov Chain of}
%        \label{figure:2-dp-mc2}
    \end{subfigure}
    \caption{Markov Chain of Double Surveys}
    \label{figure:2-dp-mc}
\end{figure}



We consider the formula $\dpriv{\ln 9}{0} (\X (Y\wedge \X Y))$. A path satisfies $\X (Y\wedge \X Y)$ if the second state satisfies $Y$ and the third state satisfies $Y$ as well. We verify that this formula is satisfied  for state $+$ and $-$. Moreover, the bound $\epsilon=\ln 9$ is tight, since the probability of satisfying $\X (Y\wedge \X Y)$ from states $+$ and $-$ are $\frac{9}{16}$ and $\frac{1}{16}$, respectively. Finally, the formula $\wedge_{a_1,a_2}\dpriv{\ln 9}{0} (\X (a_1\wedge \X a_2))$ corresponds to the desired compositional property for the model, where $a_1,a_2$ range over all atomic propositions $\{Y,N\}$.

We consider two other formulae for comparision: 
(a) Formula $\dpriv{\ln 3}{0} (\X \X Y)$. In this case we donot have privacy lost, even though we are querying twice: the reason is that the output of the first query is not utilized at all. It is easy to verify that it is indeed satisfied by $+$ and $-$.
(b) Formula $\dpriv{\ln 3}{0} (\X (Y\wedge \dpriv{\ln 3}{0}(\X Y)))$. This is a nested $\dpCTL$ formula, where the inner state formula $\dpriv{\ln 3}{0}(\X Y)$ states the one step differential privacy. Observe the inner formula is satisfied by all neighbours. Thus, the outer formula also has no privacy lost.

\paragraph{Tighter Privacy Bounds for Compositional Properties.}
One of the advantage of applying model checking is that we may get tighter bounds for compositional properties. Consider our survey example, and the property $\dpriv{0}{.5} (\X Y)$. Obviously, it holds in states $+$ and $-$. A careful check ensures that one cannot decrease the value of $\epsilon_1=0$ or $\delta_1=.5$ without increasing the other one.


Now we consider the formula $\dpriv{\epsilon_2}{\delta_2} (\X (Y\wedge \X Y))$ under the model in Figure~\ref{figure:2-dp-mc}. Applying the compositional theorem, one can choose $\epsilon_2=2\epsilon_1=0$ and $\delta_2=2\delta_1=1$. However, we can check easily that one gets better privacy parameter $(0,0.5)$ or $(-\infty,\frac{9}{16})$. \bw{use $-\infty$ instead of $\ln 0$. is it okay?}

\paragraph{Reachability Privacy.}
For our threshold \lz{to do} model, we have the differential privacy property $\wedge_{i=1}^{c} \dpriv{\epsilon}{0}(\X^k \top) \wedge \dpriv{\epsilon}{0}(F \top)$. Here $\dpriv{\epsilon}{0}(F \top)$ is a reachability privacy: it states that related states should have similar probabilities of eventually reaching $\top$.

\paragraph{Infinite Differential Privacy.}
Due to the power of linear temporal logic, our logic $\dpCTLstar$ generalizes differential privacy to infinite
behaviors naturally. Classical differential privacy discusses properties about
randomized terminating mechanisms. Sophisticated mechanisms for
streaming data have also been discussed in social networks and
internet of things. Such mechanisms inherently have infinite
computation. Their privacy guarantees are approximated by those of the
prefixes of computation.
With $\dpCTLstar$, the subformula $\phi$ in $\dpriv{\epsilon}{\delta} \phi$
can be an arbitrary $\dpCTLstar$ path formula. One can specify
$(\epsilon, \delta)$-closedness over infinite behaviors naturally. 

We consider the Bluetooth example from the PRISM benchmarks \cite{prism-bluetooth-page}. The model is a 
Markov chain, and it ...

For instance,
$\dpriv{\epsilon}{\delta} (\G   \F \mathit{discovered})$ 
specifies states having $(\epsilon, \delta)$-close
infinitely often $\mathit{discovered}$ behaviors to its neighbors.

