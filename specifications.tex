In this section we describe what type of properties in the differential privacy context can be expressed using $\dpCTL$.

\paragraph{Standard Differential Privacy.}
The formula $\dpriv{3}{0} (\X Y)$ holds in state $s$ if for  $t$ with $sNt$, the probability of satisfying $\X Y$ from $s$ and $t$ satisfy:
$\myPr{s}{K}{N}{\X Y}\le 3\myPr{t}{K}{N}{\X Y}$. The formula $\dpriv{3}{0} (\X Y)\wedge \dpriv{3}{0} (\X N)$ thus specifies the differential privacy property for states $+$ and $-$ in Example~\ref{exa:survey}.

Further, consider the Markov chain in
Figure~\ref{figure:geometric-mechanism-markov-chain}. Assume
$L(out_i) = \{ out_i \}$ and $\Delta (f) = 1$. Then $in_k$ and $in_l$ are
neighbors iff $| k - l | \leq 1$\lz{put above sentence to the def of the model}. Define the $\dpCTL$ formula
$\psi = \dpriv{.5}{0} (\F out_0) \wedge \dpriv{.5}{0} (\F out_1) \wedge
\cdots \wedge \dpriv{.5}{0} (\F out_5)$\lz{I think we better use $\X$ operator here ;)}. If the state $in_k$ satisfies
$\psi$ for $k = 0, \ldots, 5$, then the $.5$-geometric mechanism is
$(0.5, 0)$-differentially private.


\paragraph{Compositional Properties.}
Compositional aspect is one of the building block for differentially private algorithms. According to the compositional theorem \cite{}, for $(\epsilon_1,0)$-differentially private algorithm $M_1$ and
$(\epsilon_2,0)$-differentially private algorithm $M_2$, their combination $M$ defined by $M(x)=(M_1(x), M_2(x))$ is $(\epsilon_1\epsilon_2,0)$-differentially private. The degradation is rooted in the repeated computation. To illustrate this property, we consider the extended example below \lz{here example of the survey example, expanded twice: we should discuss the graph before it is drawn...}

We consider the formula $\dpriv{9}{0} (\X (Y\wedge \X Y))$. A path satisfies $\X (Y\wedge \X Y)$ if the second state satisfies $Y$ and the third state satisfies $Y$ as well. We verify that this formula is satisfied  for state $+$ and $-$. Moreover, the bound $\epsilon=9$ is bound, since \lz{tbd once graph is given}. Finally, the formula $\wedge_{a_1,a_2}\dpriv{9}{0} (\X (a_1\wedge \X a_2))$ corresponds to the desired compositional property for the model, where $a_1,a_2$ range over all atomic propositions $\{Y,N\}$.

We consider another simpler formula for comparision: $\dpriv{3}{0} (\X \X Y)$. In this case we donot have privacy lost, even though we are querying twice: the reason is that the output of the first query is not utilized at all. It is easy to verify that it is indeed satisfied by $+$ and $-$.

\paragraph{Tighter Privacy Bounds for Compositional Properties}
One of the advantage of applying model checking is that we may get tighter bounds for compositional properties. Consider our survey example, and the property $\dpriv{1}{.5} (\X Y)$. Obviously, it holds in states $+$ and $-$. A careful check ensures that one cannot decrease the value of $\epsilon_1=1$ or $\delta_1=.5$ without increasing the other one.


Now we consider the formula $\dpriv{\epsilon_2}{\delta_2} (\X (Y\wedge \X Y))$ under the model in ???. Applying the compositional theorem\lz{Bow-Yaw, as we only refer to the book, I wonder whether using the notion $e^\epsilon$ is easier to explain?}, one can choose $\epsilon_2=\epsilon_1^2=1$ and $\delta_2=2\delta_1=1$. However, we can check easily that one can get better privacy parameter $(1,0.5)$ or $(0,\frac{9}{16})$.


\paragraph{Infinite ...}
Note that $\dpCTL$ generalizes differential privacy to infinite
behaviors naturally. Classical differential privacy discusses properties about
randomized terminating mechanisms. Sophisticated mechanisms for
streaming data have also been discussed in social networks and
internet of things. Such mechanisms inherently have infinite
computation. Their privacy guarantees are approximated by those of the
prefixes of computation.
With \dpCTL, the subformula $\phi$ in $\dpriv{\epsilon}{\delta} \phi$
can be an arbitrary $\dpCTL$ path formula. One can specify
$(\epsilon, \delta)$-closedness over infinite behaviors naturally. For instance,
$\dpriv{\epsilon}{\delta} (\G \PJ{[1,1]} (\F \mathit{high}))$ \lz{now we can use $G F high$}
specifies states having $(\epsilon, \delta)$-close
infinitely often $\mathit{high}$ behaviors to its neighbors.
